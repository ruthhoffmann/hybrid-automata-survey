We will first discuss whether it is possible to abstract a given PHA. For that we introduce the probabilistic equivalent bisimulation relation, and we then discuss out several classes which can be abstracted to a PTS.

\begin{defi}[PHA as PTS]
\label{def:pha2pts}
The semantics of a PHA $\mathcal{H}=(V,L,A,\Init,\Inv,E,f,t)$ can be described in terms of a probabilistic transition system as follows; $\mathcal{P}_H=(Q,Q_{0},L,A,t)$ is the underlying transition system if
\begin{itemize}
    \item $Q=L\times\val(V)$
    \item $Q_{0} = \Init$
    \item $A=A\cup\mathbb{R}$
    \item $t=\bigcup_{\delta\in\mathbb{R}}t_{\delta}\cup \bigcup_{a\in A}t_{a}$ for $a\in A$ and $\delta\in\mathbb{R}_{\geq 0}$ with
    \begin{itemize}
        \item $t_{\delta}\subseteq Q\times\mathbb{R}_{\geq0}\times\mu(Q)$, is the largest set such that $((\ell,v),\delta,\{(\ell',v')\mapsto1\})\in t_{\delta}$ implies that $\ell=\ell'$ and $v'\in \Inv(\ell)$.
        \item $t_{a}\subseteq Q\times A \times\mu(Q)$, is the largest set of transitions such that $((\ell,v),a,p)\in t_{a}$ implies that there exists a probabilistic edge $e=(\ell,a,\ell')\in E$ with $t(\ell,a)=p_{1}(\ell')$ such that $v\in g(e)$ and there exists $[v_{1},\ldots,v_{n}]\in Bundle(v,p)$ such that for each $(\ell',v')\in Q$ \[p_{1}(\ell',v') = \sum_{1\leq i\leq n \text{s.t.} v'=v_{i}} p(\ell')\]
    \end{itemize}
\end{itemize}
\end{defi}

To define probabilistic simulation and bisimulation, we need to define what it means for distributions to be equivalent.

\begin{defi}[Weight function \cite{Jonsson1991}]
Let $R\subseteq S_{1} \times S_{2}$ be a relation between the sets $S_{1},S_{2}$ and $p_{1},p_{2}$ be distributions such that $p_{1}\in\mu(S_{1})$ and $p_{2}\in\mu(S_{2})$. A \emph{weight function}
 $w:S_{1}\times S_{2}\rightarrow [0,1]$ with respect to $R$ is a function such that for all $s_{1}
 \in S_{1}$, $s_{2}\in S_{2}$
 \begin{itemize}
     \item if $w(s_{1},s_{2})>0$ then $(s_{1},s_{2})\in R$ and
     \item $\sum_{s'\in S_{2}} (s_{1},s') = p_{1}(s_{1})$ and $\sum_{s'\in S_{1}} (s',s_{2}) = p_{2}(s_{2})$.
 \end{itemize}
\end{defi}

If there exists a weight function for $p_{1}\in\mu(S_{1})$, $p_{2}\in\mu(S_{2})$ with respect to $R\subseteq S_{1}\times S_{2}$, we write $p_{1} R p_{2}$.

\begin{defi}[Probabilistic Simulation \cite{Jonsson1991,Segala1995}]
A \emph{simulation} of a probabilistic transition system $\mathcal{P} =(Q,Q_{0},L,A,t)$ is a preorder $\sim^{P}\subset Q\times Q$ such that for each $q_{1}\sim q_{2}$:
\begin{itemize}
    \item $L(q_{1}) = L(q_{2})$ and
    \item if $t(q_{1})=(a,p_{1})$ then $t(q_{2})=(a,p_{2})$ for some $p_{2}$ such that $p_{1} R p_{2}$.
\end{itemize}
\end{defi}


\begin{defi}[Probabilistic Bisimulation \cite{Larsen1989,Segala1995}]
A \emph{bisimulation} of a probabilistic transition system $\mathcal{P}=(Q,Q_{0},L,A,t)$ is an equivalence relation $\sim_{B}^{P}\subseteq Q\times Q$ such that for each $q_{1}\sim q_{2}$,
\begin{itemize}
    \item $L(q_{1})=L(q_{2})$
    \item if $t(q_{1})=(a,p_{1})$ then $t(q_{1})=(a,p_{1})$ for some $p_{2}$ such that $p_{1} R p_{2}$ and
    \item if $t(q_{2})=(a,p_{2}')$ then $t(q_{1})=(a,p_{1}')$ for some $p_{1}'$ such that $p_{1}' R p_{2}'$.
\end{itemize}
\end{defi}

\begin{prop}[\cite{Sproston2001}]
Let $\mathcal{P} =(Q,Q_{0},L,A,t)$ be a probabilistic transition system. Two states $q_{1},q_{2}\in Q$ are bisimilar if and only if there exists an equivalence relation $R$ on the set of states such that $(q_{1},q_{2})\in Q$ and for all states $q,q'\in Q$ such that $(q,q')\in R$ we have
\begin{itemize}
    \item $L(s)=L(s')$ and
    \item if $t(s)=(a,p)$ then $t(s')=(a,p')$ for some $p'$ such that $p[C] = p'[C]$ for all equivalence classes $C\in S/_{R}$.
\end{itemize}
\end{prop}


\subsection{Can we abstract a given Probabilistic Hybrid Automaton?}
To find the abstract probabilistic transition system of a PHA we need to see whether there is an equivalent HA which can be abstracted into a transition system. For that we first construct a HA which corresponds to a given PHA. Let $\mathcal{H}=(V,L,A,\Init,\Inv,E,f,t)$ be a probabilistic hybrid automaton. We can say that a hybrid automaton is a PHA where each probabilistic transition $p\in t(\ell,a)$ has $p=1$.

The HA induced by the PHA $\mathcal{H}=(V,L,A,\Init,\Inv,E,f,t)$ is $ind(\mathcal{H})=(V,L,A\times \mu(L)\times r,\Init,\Inv,E,f,ind(t))$ where we define $ind(t)$ to be the smallest set of transitions such that if $t(\ell,a)=p$ with $e=(\ell,a,\ell')$ with $g(e)$ and $r(e)$, then for each $\ell'\in support(p)$ there exists $ind(t(\ell,a))=1:(\ell')$.
The set of actions now consists of a tuple of the action, a distribution over the locations and the reset function \cite{Sproston2014}.


\begin{prop}[\cite{Sproston2014}]
Let $\sim$ be a bisimulation on the transition system underlying $ind(\mathcal{H})$. Then $\sim$ is a probabilistic bisimultation on the PTS underlying $\mathcal{H}$.
\end{prop}

\begin{lem}[\cite{Sproston2000}]
Let $\mathcal{M}$ be a probabilistic multisingular automaton then Region equivalence $\equiv^{R}$ is a finite bisimulation of the time-abstract probabilistic transition system $S_\mathcal{M}$.
\end{lem}


\begin{lem}[\cite{Sproston2000}]
Let $\mathcal{O}$ be a probabilistic o-minimal hybrid automata then $\mathcal{O}$ has a finite bisimulation quotient.
\end{lem}

\begin{cor}[\cite{Sproston2000}]
The PBTL model checking problems for probabilistic multisingular automata and probabilistic o-minimal hybrid automata are decidable.
\end{cor}

\subsubsection{Abstracting PTAs}
\label{sec:dpta}
The symbolic model checking and abstractions of probabilistic timed automata have been studied extensively. \cite{Sproston2001} uses region graphs as the abstraction of PTA. A \emph{region graph} is a PTS where the states are equivalence classes of the states of a PTA, based on their involvement in properties. \cite{Kwiatkowska2007} uses symbolic algorithms to model check the properties and quantitative abstraction refinement to abstract the PTA. This abstraction is described in \cite{Kattenbelt2010}. A survey of techniques used for verification of properties in PTA can be found in \cite{Norman2013}.

%
% \comm{the rest of this section will most probably be useless unless i want to just copy and paste some thms together.}
%
% Let us restrict our definition of probabilistic timed automata (Definition \ref{def:PTA}) to contain transitions with discrete distributions and tightened guards.
%
% We also assume that our the set of continuous variables consist of the union of continuous clocks $V$ and the formula clocks $Z$.
%
% \begin{defi}[Tightened Guard \cite{Sproston2001}]
% For a probabilistic timed automaton $\mathcal{T}$, any location $\ell\in L$ and distribution $p\in t(\ell,a)$ with guard $g(e)$, where $e=(\ell,a,\ell')$ the \emph{tightened guard} $g'(e)$ is
% \[
%     g'(e) = g(e) \cap \Inv(\ell) \bigcap_{\ell'\in support(p)} [r(e)]\Inv(\ell')
% \]
% where $[r(e)]\Inv(\ell')$ denotes the set of valuations which lie in $\Inv(\ell')$ when the variables are reset according to $r(e)$.
% \end{defi}
%
% This tightened guard assures that the discrete transitions of the hybrid automaton are defined to admissible states of $\mathcal{T}$.
%
% \begin{defi}[Admissible State]
% A state of $\mathcal{T}$ is the pair $(\ell,x)\in L\times\val(V)$ and is \emph{admissible} if $x\in \Inv(\ell)$.
% \end{defi}
%
% We can now define the probabilistic transition system with respect to a probabilistic timed automaton, containing these restrictions.
%
% \begin{defi}[Semantics of a probabilistic timed automaton \cite{Sproston2001}]
% Let $\mathcal{T}=(V,L,A,\Init,\Inv,E,f,t)$ be a probabilistic timed automaton. Then the probabilistic transition system $\mathcal{P}_{T}=(Q,Q_{0},L,A,t)$ of $\mathcal{T}$ is defined as
% \begin{itemize}
%     \item $Q=L\times \val(V)$ is the set of states
%     \item $q_{0} = \Init$ is the set of initial states
%     \item $L:S\rightarrow 2^{AP}$ is the labelling function which assigns to states atomic propositions
%     \item $A=\{p : p\in t(\ell,a),\ell\in L, a\in A\}\cup\val(V)$ is the set of actions
%     \item for each state $q=(\ell,x)\in Q$ let $t_{P}(q,a)=cont(q,a)\cup disc(q,a)$ be the smallest set of distributions such that
%     \begin{itemize}
%         \item for each $\delta\in\mathbb{R}_{\geq0}$ there exists $(\delta,\mathcal{D}((\ell,x+\delta),a))\in cont(q,a)$ if and only if $x+\delta\in \Inv(\ell)$, where $\mathcal{D}$ denotes the Dirac distribution;
%         \item for each $e=(\ell,a,\ell')\in E$ of $\mathcal{T}$ there exists $(p,\tilde{p})\in disc((\ell,x),a)$ if and only if $p\in t(\ell,a)$, $x\in g(e)$ and $\tilde{p}\in\mu(Q)$ is such that for any $(\ell',y)\in Q$
%         \[
%             \tilde{p}(\ell',y) = \sum_{\substack{r(e)\in(\mathbb{N}\cup\{\bot\})^{n}\& \\y=x[r(e)]}}p(\ell')
%         \]
%         where $y=x[r(e)]$ indicates that $y$ is equals to the valuation of the variables after being reset.
%     \end{itemize}
% \end{itemize}
% \end{defi}
%
% However the transition system $\mathcal{P}_{T}$ of a PTA $\mathcal{T}$ is still infinite. To be able to model check the PTA reliably we need to induce a probabilistic transition system that has a finite number of states.
%
% \begin{defi}[Clock Equivalence \cite{Alur1994}]
% Let $\mathcal{T}$ be a PTA, $\phi$ a PTCTL formula and $c=c_max(\mathcal{T},\phi)$ be the maximal constant that any system clock $x\in V\cup Z$ is compared to in any of the invariant or guard conditions of $\mathcal{T}$ or in a zone subformula of $\phi$. For the $x,y\in\val(V\cup Z)$ we have that $x$ and $y$ are clock equivalent $x\equiv^{c}y$ if and only if
% \begin{itemize}
%     \item for all variables $i\in V\cup Z$ either $\lfloor x_{i}\rfloor = \lfloor y_{i}\rfloor$ or $x_{i}>c$ and $y_{i}>c$ and
%     \item for each variable $i,j\in V\cup Z$ either $frac(x_{i}-x_{j}) = frac(y_{i}-y_{j})$ or both $x_{i}-x_{j}>x$ and $y_{i}-y_{j}>c$,
% \end{itemize}
% where $\lfloor t \rfloor$ is the integer part of $t$ and $frac(t)$ is the fractional part, such that $t=\lfloor t \rfloor + frac(t)$.
% \end{defi}
%
% \begin{defi}[Region Equivalence]
% Let $\mathcal{T}$ be a probabilistic timed automaton, and $\phi$ a PTCTL formula. Two states $(\ell,x),(\ell',y)\in Q$ of the probabilistic transition system $\mathcal{P}_{T}$ are \emph{region equivalent} $(\ell,x) \equiv_{reg} (\ell',y)$ if and only if $\ell=\ell'$ and $x\equiv^{c} y$.
% We denote by $Reg_{T}^{\phi}(V\cup Z)$ the set of all regions of the PTA $\mathcal{T}$ and the PTCTL formula $\phi$.
% \end{defi}
%
%
% \begin{defi}[Probabilistic Region Graph]
% The \emph{probabilistic region graph} is the probabilistic transition system $[\mathcal{P}_{T}]=([Q],[Q_{0}],[L],[A].[t])$ defined as follows
% \begin{itemize}
%     \item $[Q] = Reg_{T}^{\phi}(V\cup Z)$, such that the set of regions is the state set;
%     \item $[Q_0]\subseteq [Q]$ the set of initial states is the set of regions of the form $(\Init, [0])$;
%     \item $[L]:[Q]\rightarrow 2^{[AP]}$ the labelling function is defined as
%     \[
%     [L](\ell,x) = L(\ell)\cup\{\pi_{\zeta}\in [AP] : x\in \zeta, \zeta\in sub(\phi)\cap Zone(V\cup Z)\}
%     \]
%     where $[AP]=AP\cup\{\pi_{\zeta}: \zeta\in sub(\phi)\cap Zone(\mathcal{T},\phi)$;
%     \item $[A] = A \cup \{\theta_{p} : p\in t(\ell,a), (\ell,a,\cdot)\in E\}\cup\{\tau\}$;
%     \item $[t]:[Q]\times[A]\rightarrow \mu(Q)$ is defined as, for each $(\ell,x)\in[Q]$  $[t]((\ell,x),a)=[Time]((\ell,x),a)\cup [Disc]((\ell,x),a)$ be the smallest set of distributions such that
%         \begin{itemize}
%             \item if there exists a time successor $x'$ of $x$ such that $x'\in \Inv(\ell)$ then $[Time]((\ell,x),a)=\{\tau,\mathcal{D}((\ell,x'),a)\}$ otherwise $[Time]((\ell,x),a)=\emptyset$.
%             \item there exists $\bar{p}\in [Disc]((\ell,x),a)$ if and only if $\exists p\in t(\ell,a)$ and $x\in g(\ell,a,\cdot)$ and where $\bar{p}$ is such that for any region $(\ell',y)\in[Q]$
%             \[
%                 \bar{p}(\ell',y)=\sum_{v\in V \& y=x[v=0]} p(v',X).
%             \]
%         \end{itemize}
% \end{itemize}
% \end{defi}
%
%
% \begin{defi}[Discrete Progress]
% Let $\mathcal{T}$ be a PTA and $\phi$ a PTCTL formula. For $[\mathcal{P}_{T}]$ the path
% \[
% w = (\ell_{0},x_{0}) \overset{\sigma_{0},p_{0}}{\rightarrow} (\ell_{1},x_{1}) \overset{\sigma_{1},p_{1}}{\rightarrow} \cdots
% \]
% exhibits \emph{discrete progress} if $\forall i\in\mathbb{N}$, $\exists j > i$ such that $\sigma_{j}\neq\tau$.
% \end{defi}
%
%
% \begin{defi}[Zero Equivalence Class]
% For $x_{i}\in V \cup Z$ the equivalence class $[\alpha]$ is said to be \emph{$x_{i}$-zero} if and only if $a_{i}=0$ for each $a_{i}\in[\alpha]$. We say that a region is $(\ell,\alpha)$ is $x_{i}$-zero if $\alpha$ is $x_{i}$-zero.
% \end{defi}
%
% \begin{defi}[Unbounded Equivalence Class]
% For $x_{i}\in V\cup Z$ the equivalence class $[\alpha]$ is said to be \emph{$x_{i}$-unbounded} if and only if $a_{i}>c_{max}(\mathcal{T},\phi)$ for each $a_{i}\in[\alpha]$. We say that a region is $(\ell,\alpha)$ is $x_{i}$-unbounded if $\alpha$ is $x_{i}$-unbounded.
% \end{defi}
%
% \begin{defi}[Time Progress]
% Let $\mathcal{T}$ be a PTA and $\phi$ a PTCTL formula. The path $w$ of $[\mathcal{P}_{T}]$
% \[
%  w = (\ell_{0},x_{0}) \overset{\sigma_{0},p_{0}}{\rightarrow} (\ell_{1},x_{1}) \overset{\sigma_{1},p_{1}}{\rightarrow} \cdots
% \]
% is
% \begin{description}
%     \item[possibly progressive] if and only if for each clock $y\in V\cup Z$ either
%     \begin{itemize}
%         \item for every $i\in\mathbb{N}$ $\exists j \geq i$ such that $x_{j}$ is an $y$-zero class or
%         \item $\exists i\in\mathbb{N}$, such that $\forall j\geq i$ $x_{j}$ is an $y$-unbounded class.
%     \end{itemize}
%     \item[zero] if and only if there exists a block $y\in V$ and $i\in\mathbb{N}$ such that for all $j > i$ $x_{j}$ is $y$-zero.
% \end{description}
% The path $w$ exhibits $\emph{time progress}$ if and only if it is possibly progressive and not zero.
% \end{defi}
%
%
% \begin{defi}[Divergent Path]
% Let $\mathcal{T}$ be a PTA and $\phi$ a PTCTL formula. For the probabilistic region graph $[\mathcal{P}_{T}]$ a path $w\in Path^{[P_{T}]}$ is \emph{divergent} if and only if it exhibits discrete progress and time progress.
% \end{defi}
%
% \begin{defi}[Divergent Adversary]
% Let $\mathcal{T}$ be a PTA and $\phi$ a PTCTL formula. An adversary $A$ of the probabilistic region graph $[\mathcal{P}_{T}]$ is \emph{divergent} if and only if
% \[
% prob^{B}(\{w\in Path^{B} : w \text{ is divergent } \}) = 1.
% \]
% % Let $[Div]$ be the set of divergent adversaries of $[\mathcal{P}_{T}]$.
% \end{defi}
%
% \begin{prop}[\cite{Sproston2001}]
% Let $\mathcal{T}$ be a PTA and $\phi$ a PTCTL formula. Then a state $(\ell,x)\in S$ of the PTS $\mathcal{P}_{T}$ satisfies $\phi$ over divergent adversaries if and only if the region $(\ell,[x])\in [\mathcal{P}_{T}]$ of the probabilistic region graph $[\mathcal{P}_{T}]$ satisfies the $PBTL$ formula $\Phi$ interpreted over divergent adversaries, where $\Phi$ is derived from $\phi$.
% \end{prop}
%
%
% \begin{ex}
% Running example goes here
% \end{ex}


\subsection{Running Example}
We will have a running example of a probabilistic HA which will illustrate the techniques introduced in the upcoming sections. A general definition of this automaton is given here; restrictions on the various components will be introduced when necessary.
\begin{ex}
    \label{ex:abs}
We look at a simplistic example of a unmanned aerial vehicle (UAV) taking off to a specific height $h$, moving forward a given distance $l$ and then landing again. The UAV can move forward with two different speeds and when landing there is a possibility that it might crash instead. One of the reasons why this model is simplistic is that we assume no acceleration in the modes.

Note that this example is a rectangular probabilistic hybrid automaton.
\begin{figure}[H]
     \begin{center}
         \begin{tikzpicture}[node distance=3.1cm, every state/.style={rectangle}]
             \node [initial, state] (s0) {$\ell_{0}$\\$\dot{x}=0$\\$\dot{z}=0$};
             \node [state] (s1) [right of=s0] {$\ell_{1}$\\$\dot{z}=v_{z,1}$\\$z\leq h$};
             \node [state] (s2) [above right of=s1,yshift=-6mm,xshift=5mm] {$\ell_{2}$\\$\dot{x}=v_{x,1}$\\$0\leq\dot{z}\leq v_{z,2}$\\$x\leq l$\\$h-\varepsilon\leq z\leq h+\varepsilon$};
             \node [state] (s3) [below right of=s1,yshift=6mm,xshift=5mm] {$\ell_{3}$\\$\dot{x}=v_{x,2}$\\$0\leq\dot{z}\leq v_{z,2}$\\$x\leq l$\\$h-\varepsilon\leq z\leq h+\varepsilon$};
             \node [state] (s4) [below right of=s2,yshift=5mm,xshift=5mm] {$\ell_{4}$\\$\dot{z}=-v_{z}$\\$z\geq 0$};
             \node [state] (s5) [right of=s4,xshift=3mm] {$\ell_{5}$};

             \path[->] (s0) edge node [above] {1:(Take Off)} (s1)
                (s1) edge node [left,yshift=2mm] {0.6:(Move,$z=h$)} (s2)
                    edge node [left,yshift=-2mm] {0.4:(Move,$z=h$)} (s3)
                (s2) edge node [left,yshift=-1mm] {1:(Stop,$x=l$)} (s4)
                (s3) edge node [left,yshift=1mm] {1:(Stop,$x=l$)} (s4)
                (s4) edge node [above] {0.01:(Land,$z=0$,\\$x:=0,z:=0$)} (s5)
                    edge [bend left=90] node [below] {0.99:(Land,$z=0$,$x:=0,z:=0$)} (s0);
         \end{tikzpicture}
         \caption{Running example of a UAV.}
         \label{fig:runex}
     \end{center}
 \end{figure}
The different component of the model in Figure~\ref{fig:runex} are
\begin{align*}
    V & = \{x,z\} \\
    L & = \{\ell_{0},\ell_{1},\ell_{2},\ell_{3},\ell_{4},\ell_{5}\} \\
    A & = \{\text{Take Off, Move, Stop, Land}\} \\
    \Init & = \{(\ell_{0},(0,0))\} \\
    \Inv(\ell_{0}) & = \Inv(\ell_{5}) = (\mathbb{R},\mathbb{R})\\
    \Inv(\ell_{1}) & = (\mathbb{R},(-\infty,h])\\
    \Inv(\ell_{2}) & = \Inv(\ell_{3}) = ((-\infty,l],[h-\varepsilon,h+\varepsilon])\\
    \Inv(\ell_{4}) & = (\mathbb{R},[0,\infty))\\
    E & = \{e_{0}=(\ell_{0},\text{Take Off},\ell_{1}), e_{1}=(\ell_{1},\text{Move},\ell_{2}), e_{2}=(\ell_{1},\text{Move},\ell_{3}), & \\
    & e_{3}=(\ell_{2},\text{Stop},\ell_{4}), e_{4}=(\ell_{3},\text{Stop},\ell_{4}), e_{5}=(\ell_{4},\text{Land},\ell_{5}), e_{6}=(\ell_{4},\text{Land},\ell_{5}) \} \\
    r(e_{0}) & = r(e_{1}) = r(e_{2}) = r(e_{3}) = r(e_{4}) = (\mathbb{R},\mathbb{R}) \\
    r(e_{5}) & = r(e_{6}) = (0,0) \\
    g(e_{0}) & = (\mathbb{R},\mathbb{R}) \\
    g(e_{1}) & = g(e_{2}) = (\mathbb{R},h) \\
    g(e_{3}) & = g(e_{4}) = (l,\mathbb{R}) \\
    g(e_{5}) & = g(e_{6}) = (\mathbb{R},0) \\
    f(\ell_{0}) & = f(\ell_{5}) = \{\dot{x} = 0,\dot{z} = 0 \} \\
    f(\ell_{1}) & = \{\dot{x} = 0,\dot{z} = v_{z,1}\} \\
    f(\ell_{2}) & = \{\dot{x} = v_{x,1},\dot{z} = [0,v_{z,2}]\} \\
    f(\ell_{3}) & = \{\dot{x} = v_{x,2},\dot{z} = [0,v_{z,2}]\} \\
    f(\ell_{4}) & = \{\dot{x} = 0,\dot{z}=-v_{z,1}\} \\
    t(\ell_{0},\text{Take Off}) & = \{1: \ell_{1}\} \\
    t(\ell_{1},\text{Move}) & = \{0.6: \ell_{2}, 0.4: \ell_{3}\} \\
    t(\ell_{2},\text{Stop}) & = t(\ell_{3},\text{Stop}) = \{1: \ell_{4}\} \\
    t(\ell_{4},\text{Land}) & = \{0.99: \ell_{0}, 0.01:\ell_{5}\}
\end{align*}
\end{ex}
%\subsection{Abstracting (discrete) Probabilistic Timed Automata}



\subsection{Translating between Different Types of Probabilistic Hybrid Automata}
\label{sec:phatrans}
We can translate between different types of PHAs to reach a PTA which we know can be verified.
The chain of translations goes rectangular to multisingular to stopwatch to timed. Further details on these translations can be found in \cite{Sproston2001}.

In Section~\ref{sec:dpta} we mentioned different ways which are used to verify a PTA.
Using these translations, we can turn the given type of PHA to a PTA, which then can be verified. Note that the PHA has to be initialised to be able to apply the abstraction. This means that the set of initial states contains a single element. Our running example (Example~\ref{ex:abs}) is initialised.

%%%%%%%%%%%%%%%%%%%%%%%%%%%%%%%%%%%%%%%%%%%%%%%%%%%%%%%%%%%%%%%%%%%%%%%%%%%%%%%%%%%%%%%%%%%%%%%%%%%
%%%%%%%%%%%%%%%%%%%%%%%%%%%%%%%%%%%%%%%%%%%%%%%%%%%%%%%%%%%%%%%%%%%%%%%%%%%%%%%%%%%%%%%%%%%%%%%%%%%
%%% PRHA to PMSHA
%%%%%%%%%%%%%%%%%%%%%%%%%%%%%%%%%%%%%%%%%%%%%%%%%%%%%%%%%%%%%%%%%%%%%%%%%%%%%%%%%%%%%%%%%%%%%%%%%%%
%%%%%%%%%%%%%%%%%%%%%%%%%%%%%%%%%%%%%%%%%%%%%%%%%%%%%%%%%%%%%%%%%%%%%%%%%%%%%%%%%%%%%%%%%%%%%%%%%%%
\subsubsection{Rectangular to Multisingular}
The translation from a rectangular to a multisingular PHA is dependent on the maximal and minimal behaviour of the continuous variables, as we are turning the intervals of the flow functions into points. For that new continuous variables are introduced in the multisingular PHA, which represent the extremal flow behaviour, and all conditions in the PHA are adjusted accordingly.

Let $\mathcal{R} = (V_{R},L,A,\Init_{R},\Inv_{R},E_{R},f_{R},t_{R})$ and $\mathcal{M} = (V_{M}, L, A, \Init_{M}, \Inv_{M}, E_{M}, f_{M}, t_{M})$ be a rectangular and a multisingular PHA respectively. We can construct $\mathcal{M}$ from $\mathcal{R}$ by the following steps
\begin{description}
    \item[Continuous Variables] Let $V_{R}=\{x_{1},\ldots,x_{n}\}$ and $V_{M}=\{y_{1},\ldots,y_{2n}\}$ where the $l(i)$-th variable $y_{l(i)}$ represents the lower bound of $x_{i}$ and $y_{u(i)}$ is the upper bound of $x_{i}$, and $l(i)=2i-1,\ u(i)=2i$.

    \item[Initial States] For each $\ell\in L$ let $\Init_{M}(\ell)(y_{l(i)})=\Init_{M}(\ell)(y_{u(i)})=\Init_{R}(\ell)(x_{i})$.

    \item[Invariant Conditions] For each $\ell\in L$ and each $x_{i}\in V_{R}$ if $\Inv_{R}(\ell)(x_{i})=[l,u]$ then
    \begin{align*}
    \Inv_{M}(\ell)(y_{l(i)}) & = [l,\infty), \\
    \Inv_{M}(\ell)(y_{u(i)}) & = (-\infty,u].
    \end{align*}
    \item[Flow] For each $\ell\in L$ and $x_{i}\in V_{R}$ if $f_{R}(\ell)=\dot{x}_{i}\in[l,u]$ then $f_{M}(\ell)(y_{l(i)})= \dot{y}_{l(i)}\in[l,l]$ and $f_{M}(\ell)(y_{u(i)}) = \dot{y}_{u(i)}\in[u,u]$.

    \item[Edges] For each $\ell\in L$, $a\in A$ and $p\in t(\ell,a)$ we can derive a set of edges and distributions in $\mathcal{M}$. To uniquely identify each distribution we have to keep track of status of the reset and guard conditions in $\mathcal{R}$.
    We will assign a \emph{status} number $\{1,\ldots,4\}$ to each variable $x_{i}$ in $\mathcal{R}$ as follows, let $x_{i}$ have bounds $y_{l(i)}$ and $y_{u(i)}$, and let $[gl_{i},gu_{i}]\in g_{R}(e)(x_{i})$, for $e=(\ell,a,\ell')\in E_{R}$ then
    \begin{enumerate}
        \item $\stat(x_{i})=1$ if $y_{l(i)} < gl_{i}$ and $y_{u(i)} \leq gu_{i}$;
        \item $\stat(x_{i})=2$ if $y_{l(i)} < gl_{i}$ and $y_{u(i)} > gu_{i}$;
        \item $\stat(x_{i})=3$ if $y_{l(i)} \geq gl_{i}$ and $y_{u(i)} \leq gu_{i}$;
        \item $\stat(x_{i})=4$ if $y_{l(i)} \geq gl_{i}$ and $y_{u(i)} > gu_{i}$.
    \end{enumerate}
    We can now identify the reset conditions for $\mathcal{M}$ and mark them with the status. Let $e=(\ell,a,\ell')\in E_{R}$ with $r_{R}(e)$ and $g_{R}(e)$ being the reset and guard conditions. Let $x_{i}\in[l_{i},u_{i}]\in g_{R}(e)$ and should $x_{i}$ be
    reset on $e$ then it will be of the form $x_{i}\in[l_{i}',u_{i}']\in r_{R}(e)$.
    \begin{itemize}
        \item If $x_{i}$ is reset then $y_{l(i)}=l_{i}'\in r_{M}^{st}(e)$ and $y_{u(i)}=u_{i}'\in r_{M}^{st}(e)$ for each $st\in\{1,\ldots,4\}$.
        \item If $x_{i}$ is not reset then
            \begin{itemize}
                \item if $\stat(x_{i})=1$ then $y_{l(i)}=l_{i}\in r_{M}^{1}(e)$ and $y_{u(i)}\notin r_{M}^{1}(e)$;
                \item if $\stat(x_{i})=2$ then $y_{l(i)}=l_{i}\in r_{M}^{2}(e)$ and $y_{u(i)}=u_{i}\in r_{M}^{2}(e)$;
                \item if $\stat(x_{i})=3$ then $y_{l(i)}\notin r_{M}^{3}(e)$ and $y_{u(i)}\notin r_{M}^{3}(e)$;
                \item if $\stat(x_{i})=4$ then $y_{l(i)}\notin r_{M}^{4}(e)$ and $y_{u(i)}=u_{i}\in r_{M}^{4}(e)$.
            \end{itemize}
    \end{itemize}
    We utilise a similar system to establish the guard conditions in $\mathcal{M}$. For every $x_{i}\in V_{R}$, if $x_{i}\in[l,u]\in g_{R}(e)$ then
        \begin{itemize}
            \item if $\stat(x_{i})=1$ then $y_{l(i)}\in(-\infty,l)\in g_{M}^{1}(e)$ and $y_{u(i)}\in[l,u]\in g_{M}^{1}(e)$;
            \item if $\stat(x_{i})=2$ then $y_{l(i)}\in(-\infty,l)\in g_{M}^{2}(e)$ and $y_{u(i)}\in(u,\infty)\in g_{M}^{2}(e)$;
            \item if $\stat(x_{i})=3$ then $y_{l(i)}\in[l,u]\in g_{M}^{3}(e)$ and $y_{u(i)}\in[l,u]\in g_{M}^{3}(e)$;
            \item if $\stat(x_{i})=4$ then $y_{l(i)}\in[l,u]\in g_{M}^{4}(e)$ and $y_{u(i)}\in(u,\infty)\in g_{M}^{4}(e)$.
        \end{itemize}

    \item[Probabilistic Transition] Through the classification of the guard and reset conditions we can now calculate the probabilistic transitions. We say that we obtain $e=(\ell,a,\ell')\in E_{M}$ with $g_{M}^{st}(e_{M})$ and $r_{M}^{st}(e_{M})$ from $e=(\ell,a,\ell')\in E_{R}$ with $g_{R}(e_{R})$ and $r_{R}(e_{R})$, and denote this by $e\leadsto e^{st}$, then
    \[
    t_{M}^{st}(\ell,a)(\ell') = \sum_{\substack{t_{R}(\ell,a)(\ell')\neq0 \land\\ e\leadsto e^{st}}} t_{R}(\ell,a)(\ell')
    \]
    for each $st\in\{1,\ldots,4\}$.
\end{description}



\begin{ex}
    \label{ex:rect2ms}
Take the rectangular hybrid automaton defined in Example~\ref{ex:abs}. Using the above description we can translate that hybrid automaton to a multisingular hybrid automaton $\mathcal{M}=(V,L,A,\Init,\Inv,E,f,t)$ where the components are
\begin{itemize}
    \item $V=\{x_{l},x_{u},z_{l},z_{u}\}$
    \item $L=\{\ell_{0},\ell_{1},\ell_{2},\ell_{3},\ell_{4},\ell_{5}\}$
    \item $A=\{\text{TakeOff, Move, Stop, Land}\}$
    \item $\Init=\{(\ell_{0},(0,0,0,0))\}$
    \item The invariant conditions are
    \begin{align*}\Inv(\ell_{0})=&\Inv(\ell_{5})=(\mathbb{R},\mathbb{R},\mathbb{R},\mathbb{R})\\
        \Inv(\ell_{1})=&(\mathbb{R},\mathbb{R},\mathbb{R},(-\infty,h]), \\
        \Inv(\ell_{2})=&\Inv(\ell_{3})=(\mathbb{R},(-\infty,l],[h-\varepsilon,\infty),(-\infty,h+\varepsilon]), \\
        \Inv(\ell_{4})=& (\mathbb{R},\mathbb{R},[0,\infty),\mathbb{R})
    \end{align*}
    \item The flow conditions are
    \begin{align*}
        f(\ell_{0})=&f(\ell_{5})=\{\dot{x}_{l}=\dot{x}_u=\dot{z}_{l}=\dot{z}_{u}=0\},\\
        f(\ell_{1})=&\{\dot{x}_{l}=\dot{x}_{u}=0,\dot{z}_{l}=\dot{z}_{u}=v_{z1}\},\\
        f(\ell_{2})=&\{\dot{x}_{l}=\dot{x}_{u}=v_{x1},\dot{z}_{l}=0,\dot{z}_{u}=v_{z2}\},\\
        f(\ell_{3})=&\{\dot{x}_{l}=\dot{x}_{u}=v_{x2},\dot{z}_{l}=0,\dot{z}_{u}=v_{z2}\},\\
        f(\ell_{4})=&\{\dot{x}_{l}=\dot{x}_{u}=0,\dot{z}_{l}=\dot{z}_{u}=-v_{z1}\}
    \end{align*}
    \item The edges, reset conditions, guard conditions and probability distributions are as follows
    \begin{itemize}
        \item The edge $e_{0}=(\ell_{0},)(\text{TakeOff}, \ell_{1})$ from the original hybrid automaton is the same in $\mathcal{M}$ due to its trivial guard and reset conditions for all variables and its Dirac distribution.
        Thus in $\mathcal{M}$, $g(e_{0})=(\mathbb{R},\mathbb{R},\mathbb{R},\mathbb{R})$, $r(e_{0})=\emptyset$ and $t(\ell_{0},)(\text{TakeOff})=\{1:\ell_{1}\}$.
        \item For $(\ell_{1},\text{Move})$ we find that $\stat(x)=3$ and similarily $\stat(z)=3$, the guard and reset conditions for $x_{l},x_{u}$ are trivially induced. Then as $z$ is not reset, $z_{l}$ and $z_{u}$ will not either, and finally as $\stat(z)=3$ the guard conditions are $z_{l}=h$ and $z_{u}=h$.
        Thus $g(\ell_{1},\text{Move},\ell_{2})=g(\ell_{1},\text{Move},\ell_{3})=(\mathbb{R},\mathbb{R},h,h)$ and $r(\ell_{1},\text{Move},\ell_{2})=r(\ell_{1},\text{Move},\ell_{3})=\emptyset$.
        Based on that the probability transition function stays the same for $t(\ell_{1},\text{Move})=\{0.6:\ell_{2},0.4:\ell_{3}\}$.
        \item $(\ell_{2},\text{Stop})$ is transformed similarly as the last edge, with $\stat(x)=\stat(z)=3$. So $g(\ell_{2},\text{Stop},\ell_{4})=(l,l,\mathbb{R},\mathbb{R})$ and $r(\ell_{2},\text{Stop},\ell_{4})=\emptyset$, and $t(\ell_{2},\text{Stop})=\{1:\ell_{4}\}$.
        \item $(\ell_{3},\text{Stop})$ is translated as $(\ell_{2},\text{Stop})$, thus
        $g(\ell_{3},\text{Stop},\ell_{4})=(l,l,\mathbb{R},\mathbb{R})$ and $r(\ell_{3},\text{Stop},\ell_{4})=\emptyset$, and $t(\ell_{3},\text{Stop})=\{1:\ell_{4}\}$.
        \item For $(\ell_{4},\text{Land})$ is translated through $\stat(x)=\stat(z)=3$ thus the reset maps are $r(\ell_{4},\text{Land},\ell_{0})=r(\ell_{4},\text{Land},\ell_{5})=(0,0,0,0)$ and the guard conditions are $g(\ell_{4},\text{Land},\ell_{0})=g(\ell_{4},\text{Land},\ell_{5})=(\mathbb{R},\mathbb{R},0,0)$ and the probability transition function is $t(\ell_{4},\text{Land})=\{0.99:\ell_{0},0.01:\ell_{5}\}$.
    \end{itemize}
\end{itemize}

\end{ex}
%%%%%%%%%%%%%%%%%%%%%%%%%%%%%%%%%%%%%%%%%%%%%%%%%%%%%%%%%%%%%%%%%%%%%%%%%%%%%%%%%%%%%%%%%%%%%%%%%%%
%%%%%%%%%%%%%%%%%%%%%%%%%%%%%%%%%%%%%%%%%%%%%%%%%%%%%%%%%%%%%%%%%%%%%%%%%%%%%%%%%%%%%%%%%%%%%%%%%%%
%%% PMSHA to PSWA
%%%%%%%%%%%%%%%%%%%%%%%%%%%%%%%%%%%%%%%%%%%%%%%%%%%%%%%%%%%%%%%%%%%%%%%%%%%%%%%%%%%%%%%%%%%%%%%%%%%
%%%%%%%%%%%%%%%%%%%%%%%%%%%%%%%%%%%%%%%%%%%%%%%%%%%%%%%%%%%%%%%%%%%%%%%%%%%%%%%%%%%%%%%%%%%%%%%%%%%
\subsubsection{Multisingular To Stopwatch}
From a multisingular PHA $\mathcal{M} = (V,L,A,\Init_{M},\Inv_{M},E,f_{M},t)$ we are going to construct a probabilistic stopwatch automaton $\mathcal{W} = (V,L,A,\Init_{W},\Inv_{W},E,f_{W},t)$. For this abstraction to work, we need to introduce a scaling factor for each variable at each location.

\begin{defi}[Scaling Factor]
    In $\mathcal{M}$ for all $\ell\in L$, $x_{i}\in V$ of $\mathcal{M}$ if $f_{M}(\ell)=\dot{x}\in[k,k]$ for some $k\in\mathbb{Z}$ then the \emph{scaling factor} of $x_{i}$ in $\ell$ is $k_{i}^{\ell}$, where $k_{i}^{\ell}=k$ if $k\neq0$ and $k_{i}^{\ell}=1$ if $k=0$.
\end{defi}
The scaling factor will help with skewing the flow functions to represent clocks (have flow equal to $1$). The construction of $\mathcal{W}$ is done as follows
\begin{description}
    \item[Initial States] Let $\ell\in L$ such that $\Init_{M}(\ell)\neq\emptyset$, and $\Init_{M}(\ell)\in\mathbb{Z}^{n}$. For every $x_{i}\in V$ if $x_{i}\in[k,k]\in \Init_{M}(\ell)$ for some $k\in\mathcal{Z}$ then let
    $x_{i}\in[\frac{k}{k_{i}^{\ell}},\frac{k}{k_{i}^{\ell}}]\in \Init_{W}(\ell)$.
    \item[Invariant Conditions] For each $\ell\in L$ and each $x_{i}\in V$ if the lower and upper bounds of $x_{i}$ in $\Inv_{M}(\ell)$ are $k_{1},k_{2}\in\mathbb{Z}$ respectively, then the lower and upper bounds of $x_{i}$ in $\Inv_{W}(\ell)$ are
    $\frac{k_{1}}{k_{i}^{\ell}},\frac{k_{2}}{k_{i}^{\ell}}$ respectively, and the limits of the interval are the same (open is mapped to open and closed is mapped to closed).
    \item[Flow] For all $\ell\in L$ and all $x_{i}\in V$ if $f_{M}(\ell)=\dot{x}_{i}\in[0,0]$ then $f_{W}(\ell)=\dot{x}_{i}\in[0,0]$ else $f_{W}(\ell)=\dot{x}_{i}\in[1,1]$.
    \item[Edge Conditions] While the edge set is the same in both automata, we need to rescale the reset and guard conditions. Let $e=(\ell,a,\ell')\in E$, then the reset conditions are derived from $r_{M}(e)(x_{i})=k\in\mathbb{Z}$ then $r_{W}(e)(x_{i})= \frac{k}{k_{i}^{\ell'}}$, for each $x_{i}\in V$. The guard conditions are scaled similarly as the invariant conditions, namely for each $x_{i}\in V$ if the lower and upper bounds of
    $x_{i}$ in $g_{M}(e)$ are $k_{1},k_{2}\in\mathbb{Z}$ respectively, then the lower and upper bounds of $x_{i}$ in $g_{W}(e)$ are $\frac{k_{1}}{k_{i}^{\ell}},\frac{k_{2}}{k_{i}^{\ell}}$ respectively, and the limits of the interval are the same (open is mapped to open and closed is mapped to closed).
\end{description}

Note that for this translation to work, the following restriction has to be added to the invariant conditions of the multisingular PHA $\mathcal{M}$. At any location $\ell\in L$ any variable with positive (negative) slope has a finite lower (upper) bound in the invariant. This restriction can be enforced without loss of generality by replacing infinite bounds with the minimal and maximal values the variable can achieve globally \cite{Olivero1994}.

\begin{defi}[[Equivalence between $\mathcal{W}$ and $\mathcal{M}$ \cite{Henzinger1995}]
Let $\mathcal{M}$ be an initialised multisingular PHA, let $\mathcal{W}$ be an initialised probabilistic stopwatch automaton derived from $\mathcal{M}$ as described above, let
 $\mathcal{P}_{M}=(Q^{M},Q^{M}_{0},A^{M},t^{M})$ and
 $\mathcal{P}_{W}=(Q^{W},Q^{W}_{0},A^{W},t^{W})$ be their probabilistic transitions systems (see Definition~\ref{def:pha2pts}) and let $\{k_{i}^{\ell}\in\mathbb{Z} : \ell\in L_{M}, x_{i}\in X\}$ be the set of scaling factors of $\mathcal{M}$. Then let $\sigma : Q^{W} \rightarrow Q^{M}$ be the
 bijection such that for each $(\ell, \val(x))\in Q^{W}$ we have $\sigma(\ell,\val(x))=(\ell,\val(y))$ where $\val(y)_{i} = k_{i}^{\ell}\val(x)_{i}$ for each $x_{i}\in X$.
 The relation $\sim_{\sigma}$ is an equivalence on $Q^{W}\cup Q^{M}$ such that $s\sim_{\sigma}s'$ if and only if $\sigma(s)=s'$ for $s\in Q^{M}$ and $s'\in Q^{W}$.
\end{defi}

\begin{prop}[\cite{Sproston2001}]
Let $\mathcal{M}$ be an initialised multisingular PHA and let $\mathcal{W}$ be the initialised stopwatch automaton constructed from $\mathcal{M}$ using the method described above. Then the equivalence relation $\sim_{\sigma}$ is a bisimulation between $\mathcal{P}_{M}$ and $\mathcal{P}_{W}$.
\end{prop}


\begin{ex}
\label{ex:ms2sw}
We are going to translate the resulting automaton from Example~\ref{ex:rect2ms} into a probabilistic stopwatch automaton. For that we need to restrict the invariant conditions of our multisingular hybrid automaton to
\begin{align*}
    \Inv(\ell_{0})=&\Inv(\ell_{5})=(\mathbb{R},\mathbb{R},\mathbb{R},\mathbb{R})\\
    \Inv(\ell_{1})=&(\mathbb{R},\mathbb{R},[0,\infty),[0,h]), \\
    \Inv(\ell_{2})=&\Inv(\ell_{3})=([0,\infty),[0,l],[h-\varepsilon,\infty),[0,h+\varepsilon]), \\
    \Inv(\ell_{4})=& (\mathbb{R},\mathbb{R},[0,h+\varepsilon],(-\infty,h+\varepsilon]).
\end{align*}

Further we need to calculate the scaling factors of the continuous variables in each location, see Table~\ref{tab:sf}. Now we can utilise these factors to construct a probabilistic stopwatch automaton from the multisingular probabilistic hybrid automaton in Example~\ref{ex:rect2ms} (with adjusted invariant conditions).
\begin{table}
\begin{center}
    \begin{tabular}{c|c|c|c|c|c|c}
                & $\ell_{0}$ & $\ell_{1}$ & $\ell_{2}$ & $\ell_{3}$ & $\ell_{4}$ & $\ell_{5}$ \\ \hline
        $x_{l}$ & $1$ & $1$ & $v_{x1}$ & $v_{x2}$ & $1$ & $1$ \\ \hline
        $x_{u}$ & $1$ & $1$ & $v_{x1}$ & $v_{x2}$ & $1$ & $1$ \\ \hline
        $z_{l}$ & $1$ & $v_{z1}$ & $1$ & $1$ & $-v_{z1}$ & $1$ \\ \hline
        $z_{u}$ & $1$ & $v_{z1}$ & $v_{z2}$ & $v_{z2}$ & $-v_{z1}$ & $1$

    \end{tabular}
    \caption{Table of scaling factors for translation of the running example.}
    \label{tab:sf}
\end{center}
\end{table}
\begin{itemize}
    \item $V=\{x_{l},x_{u},z_{l},z_{u}\}$
    \item $L=\{\ell_{0},\ell_{1},\ell_{2},\ell_{3},\ell_{4},\ell_{5}\}$
    \item $A=\{\text{TakeOff, Move, Stop, Land}\}$
    \item $\Init=\{(\ell_{0},(0,0,0,0))\}$
    \item The invariants for the locations are as above stated
     \begin{align*}
        \Inv(\ell_{0})=&\Inv(\ell_{5})=(\mathbb{R},\mathbb{R},\mathbb{R},\mathbb{R})\\
        \Inv(\ell_{1})=&(\mathbb{R},\mathbb{R},[0,\infty),[0,\frac{h}{v_{z1}}]), \\
        \Inv(\ell_{2})=&([0,\infty),[0,\frac{l}{v_{x1}}],[h-\varepsilon,\infty),[0,\frac{h+\varepsilon}{v_{z2}}]), \\
        \Inv(\ell_{3})=&([0,\infty),[0,\frac{l}{v_{x2}}],[h-\varepsilon,\infty),[0,\frac{h+\varepsilon}{v_{z2}}]), \\
        \Inv(\ell_{4})=&(\mathbb{R},\mathbb{R},[-\frac{h+\varepsilon}{v_{z1}},0],[-\frac{h+\varepsilon}{v_{z1}},\infty))
    \end{align*}
    \item $E=\{e_{0}=(\ell_{0},\text{TakeOff},\ell_{1}), e_{1}=(\ell_{1},\text{Move},\ell_{2}), e_{2}=(\ell_{1},\text{Move},\ell_{3}), e_{3}=(\ell_{2},\text{Stop},\ell_{4}), e_{4}=(\ell_{3},\text{Stop},\ell_{4}), e_{5}=(\ell_{4},\text{Land},\ell_{5}), e_{6}=(\ell_{4},\text{Land},\ell_{0})\}$ is the set of edges, which is the same as in the multisingular automaton before. The guard conditions are defined as
    \begin{align*}
        g(e_{0})= & (\mathbb{R},\mathbb{R},\mathbb{R},\mathbb{R}) \\
        g(e_{1})= & g(e_{2}) = (\mathbb{R},\mathbb{R},\frac{h}{v_{z1}},\frac{h}{v_{z1}}) \\
        g(e_{3})= & (\frac{l}{v_{x1}}, \frac{l}{v_{x1}}, \mathbb{R},\mathbb{R}) \\
        g(e_{4})= & (\frac{l}{v_{x2}}, \frac{l}{v_{x2}}, \mathbb{R},\mathbb{R}) \\
        g(e_{5})= & g(e_{6})=(\mathbb{R},\mathbb{R},0,0),
    \end{align*}
    and the reset maps are
    \begin{align*}
        r(e_{0})= & (x_{l},x_{u},\frac{z_{l}}{v_{z1}},\frac{z_{u}}{v_{z1}})\\
        r(e_{1})= & (\frac{x_{l}}{v_{x1}},\frac{x_{u}}{v_{x1}},z_{l},\frac{z_{u}}{v_{z2}}) \\
        r(e_{2})= & (\frac{x_{l}}{v_{x2}},\frac{x_{u}}{v_{x2}},z_{l},\frac{z_{u}}{v_{z2}}) \\
        r(e_{3})= & r(e_{4})=(x_{l},x_{u},-\frac{z_{l}}{v_{z1}},-\frac{z_{u}}{v_{z1}})\\
        r(e_{5})=& r(e_{6})=(0,0,0,0).
    \end{align*}
    \item We transform the flow conditions to the following
    \begin{align*}
        f(\ell_{0})=f(\ell_{5})=& \{\dot{x}_{l}=\dot{x}_{u}=\dot{z}_{l}=\dot{z}_{u}=0\} \\
        f(\ell_{1})=f(\ell_{4})=& \{\dot{x}_{l}=\dot{x}_{u}=0,\dot{z}_{l}=\dot{z}_{u}=1\} \\
        f(\ell_{2})=f(\ell_{3})=& \{\dot{x}_{l}=\dot{x}_{u}=\dot{z}_{u}=1,\dot{z}_{l}=0\}.
    \end{align*}
    \item The probabilistic transition function is the same as for the multisingular probabilistic hybrid automaton.
\end{itemize}
\end{ex}

%%%%%%%%%%%%%%%%%%%%%%%%%%%%%%%%%%%%%%%%%%%%%%%%%%%%%%%%%%%%%%%%%%%%%%%%%%%%%%%%%%%%%%%%%%%%%%%%%%%
%%%%%%%%%%%%%%%%%%%%%%%%%%%%%%%%%%%%%%%%%%%%%%%%%%%%%%%%%%%%%%%%%%%%%%%%%%%%%%%%%%%%%%%%%%%%%%%%%%%
%%% PSWA to PTA
%%%%%%%%%%%%%%%%%%%%%%%%%%%%%%%%%%%%%%%%%%%%%%%%%%%%%%%%%%%%%%%%%%%%%%%%%%%%%%%%%%%%%%%%%%%%%%%%%%%
%%%%%%%%%%%%%%%%%%%%%%%%%%%%%%%%%%%%%%%%%%%%%%%%%%%%%%%%%%%%%%%%%%%%%%%%%%%%%%%%%%%%%%%%%%%%%%%%%%%
\subsubsection{Stopwatch To Timed}
We will now translate a probabilistic stopwatch automaton $\mathcal{W} = (V,A,\Init_{W},\Inv_{W},E_{W},f_{W},t_{W})$ to a probabilistic timed automaton $\mathcal{T} = (V,A,\Init_{T},\Inv_{T},E_{T},f_{T},t_{T})$. This translation will turn all the remaining flow conditions from 0 to 1. For that we introduce a new variable $\bot$ which represents abstract time, and with that we will extend model, as follows;

\begin{description}
\item[Locations] Let $K$ be the set of integer constants of the guard, reset, initial or invariant  conditions of $\mathcal{W}$, let $K_{\bot}=K\cup\{\bot\}$. Then $L_{T}=L_{W}\times \Gamma$, where $\Gamma$ is a set of functions of the form $\gamma:V\rightarrow K_{\bot}$. For a given
$\ell\in L_{W}$ the location in $\mathcal{T}$ is of the form $(\ell,\gamma)\in L_{T}$, where for each $x_{i}\in V$ if $f_{W}(\ell)=\dot{x}_{i}=1$ then $\gamma(x_{i}) = \bot$, otherwise
$\gamma(x_{i})\in K$.
\item[Initial States] For all $(\ell,\gamma)\in L_{T}$ if for each $x_{i}\in V$ either $\gamma(x_{i}) = \bot$, or $\gamma(x_{i}) = \Init_{W}(\ell)(x_{i})$ then let $\Init_{T}(\ell,\gamma) = \Init_{W}(\ell)$ otherwise $\Init_{T}(\ell,\gamma)=\emptyset$.
\item[Invariant Conditions] For all $(\ell,\gamma)\in L_{T}$, for each $x_{i}\in V$ if
    \begin{itemize}
        \item $\gamma(x_{i}) = \bot$ then $\Inv_{T}(\ell,\gamma)(x_{i}) = \Inv_{W}(\ell)(x_{i})$.
        \item $\gamma(x_{i}) \in \Inv_{W}(\ell)(x_{i})$ then $\Inv_{T}(\ell,\gamma)(x_{i})=\mathbb{R}_{\geq 0}$
        \item $\gamma(x_{i}) \notin \Inv_{W}(\ell)(x_{i})$ then $\Inv_{T}(\ell,\gamma)(x_{i})=\emptyset$.
    \end{itemize}
\item[Edge Conditions] We derive the edge set for $\mathcal{T}$ as follows, let $e_{W}=(\ell,a,\ell')\in E_{W}$ then there exists $((\ell,\gamma), a, (\ell',\gamma'))\in E_{T}$ where for every $x_{i}\in V$ either
    \begin{itemize}
        \item $f_{W}(\ell')=\dot{x}_{i} = 1$ and $\gamma'(x_{i}) = \bot$
        \item $f_{W}(\ell)=\dot{x}_{i}=1$, $f_{W}(\ell')=\dot{x}_{i}=0$ and $\gamma'(x_{i})=r_{W}(e_{W})(x_{i})$
        \item $f_{W}(\ell)=\dot{x}_{i}=0$, $f_{W}(\ell')=\dot{x}_{i}=0$, $x_{i}$ does not get reset and $\gamma'(x_{i})=\gamma(x_{i})$
        \item $f_{W}(\ell)=\dot{x}_{i}=0$, $f_{W}(\ell')=\dot{x}_{i}=0$, $r_{W}(e_{W})(x_{i})$ exists and $\gamma'(x_{i})=r_{W}(e_{W})(x_{i})$.
    \end{itemize}
The reset maps stay the same, and are distributed onto the new edges accordingly to $\ell'\in L_{W}$. The guard conditions on an edge for a variable $x_{i}$ is the same as in the stopwatch automaton if $\gamma(x_{i})=\bot$, otherwise if $\gamma(x_{i})\in g_{W}(e)$ then let $g_{T}(e)(x_{i})=\mathbb{R}_{\geq 0}$, else $x_{i}$ is unguarded.
\item[Probability transitions] Let $t_{W}(\ell,a) = \{p_{W}^{1},\ldots,p_{W}^{l}\}$ be the probabilities from $\ell\in L_{W}$, and $t_{T}((\ell,\gamma),a) = \{p_{T}^{1},\ldots,p_{T}^{l}\}$.
Then for each $\ell'\in L_{W}$ in $(\ell,a,\ell')\in E$ there exists a $(\ell',\gamma)\in L_{T}$ according to the construction of the edges above and then $p_{T}^{j}(\ell',\gamma) = p_{W}^{j}(\ell')$, for $1\leq j \leq l$.
\end{description}

Note that $\bot$ can be seen as a abstract time passing variable.

\begin{defi}[Equivalence between $\mathcal{T}$ and $\mathcal{W}$ \cite{Henzinger1995}]
Let $\mathcal{W}$ be an initialised probabilistic stopwatch automaton, let $\mathcal{T}$ be a probabilistic timed automaton derived from $\mathcal{W}$ as described above, let
$\mathcal{P}_{W}=(Q^{W},Q^{W}_{0},A^{W},t^{W})$ and $\mathcal{P}_{T}=(Q^{T},Q^{T}_{0},A^{T},t^{T})$ be their associated probabilistic transition systems (see Definition~\ref{def:pha2pts}). Then
$\tau : Q^{T}\rightarrow Q^{W}$ is defined such that for each $((\ell,\gamma),\val(v))\in Q^{T}$, we have $\tau((\ell,\gamma),\val(x))=(\ell,\val(y))$ where $\val(x)_{i}=\val(y)_{i}$ if
$\gamma(x_{i})=\bot$ and otherwise $\val(y_{i})=\gamma(x_{i})$, for each $x_{i}\in V$. The relation $\sim_{\tau}$ is an equivalence on $Q^{T}\cup Q^{W}$ such that $q\sim_{\tau}q'$ if and only if  $\tau(q)=q'$ for $q\in Q^{T}$ and $q'\in Q^{W}$.
\end{defi}

\begin{prop}[\cite{Sproston2001}]
Let $\mathcal{W}$ be an initialised probabilistic stopwatch automaton, and let $\mathcal{T}$ be the probabilistic timed automaton derived from $\mathcal{W}$ using the method above. Then the equivalence $\sim_{\tau}$ is a bisimulation between $\mathcal{P}_{W}$ and $\mathcal{P}_{T}$.
\end{prop}


\begin{ex}
We will now translate the stopwatch automaton resulting in Example~\ref{ex:ms2sw} to a PTA.
\begin{itemize}
    \item First we calculate a preliminary set of locations. We will extend the set once the initial state and edges have been determined.
    \begin{align*}
    L=\{&(\ell_{0},(\gamma(x_{l}),\gamma(x_{u}),\gamma(z_{l}),\gamma(z_{u}))), \\
        &(\ell_{1},(\gamma(x_{l}),\gamma(x_{u}),\bot,\bot)), \\
        &(\ell_{2},(\bot,\bot,\gamma(z_{l}),\bot)), \\
        &(\ell_{3},(\bot,\bot,\gamma(z_{l}),\bot)), \\
        &(\ell_{4},(\gamma(x_{l}),\gamma(x_{u}),\bot,\bot)), \\
        &(\ell_{5},(\gamma(x_{l}),\gamma(x_{u}),\gamma(z_{l}),\gamma(z_{u})))\}
    \end{align*}
    \item $\Init=(\ell_{0},(0,0,0,0))$
    \item We can find the set of edges to be
    \begin{align*}
    E=\{e_{0}=&((\ell_{0},(0,0,0,0)),\text{TakeOff},(\ell_{1},(0,0,\bot,\bot))), \\
        e_{1}=&((\ell_{1},(0,0,\bot,\bot)),\text{Move},(\ell_{2},(\bot,\bot,\frac{h}{v_{z1}}))),\\
        e_{2}=&((\ell_{1},(0,0,\bot,\bot)),\text{Move},(\ell_{3},(\bot,\bot,\frac{h}{v_{z1}}))),\\
        e_{3}=&((\ell_{2},(\bot,\bot,\frac{h}{v_{z1}})),\text{Stop},(\ell_{4},(\frac{l}{v_{x1}},\frac{l}{v_{x1}},\bot,\bot))),\\
        e_{4}=&((\ell_{3},(\bot,\bot,\frac{h}{v_{z1}})),\text{Stop},(\ell_{4},(\frac{l}{v_{x2}},\frac{l}{v_{x2}},\bot,\bot))),\\
        e_{5}=&(\ell_{4},(\frac{l}{v_{x1}},\frac{l}{v_{x1}},\bot,\bot)),\text{Land},(\ell_{5},(0,0,0,0))),\\
        e_{6}=&(\ell_{4},(\frac{l}{v_{x1}},\frac{l}{v_{x1}},\bot,\bot)),\text{Land},(\ell_{0},(0,0,0,0))),\\
        e_{7}=&(\ell_{4},(\frac{l}{v_{x2}},\frac{l}{v_{x2}},\bot,\bot)),\text{Land},(\ell_{5},(0,0,0,0))),\\
        e_{8}=&(\ell_{4},(\frac{l}{v_{x2}},\frac{l}{v_{x2}},\bot,\bot)),\text{Land},(\ell_{0},(0,0,0,0)))\},
    \end{align*}
    then the guard conditions are
    \begin{align*}
        g(e_{0})= & (\mathbb{R},\mathbb{R},\mathbb{R},\mathbb{R}) \\
        g(e_{1})= & g(e_{2}) = (\mathbb{R},\mathbb{R},\frac{h}{v_{z1}},\frac{h}{v_{z1}}) \\
        g(e_{3})= & (\frac{l}{v_{x1}}, \frac{l}{v_{x1}}, \mathbb{R},\mathbb{R}) \\
        g(e_{4})= & (\frac{l}{v_{x2}}, \frac{l}{v_{x2}}, \mathbb{R},\mathbb{R}) \\
        g(e_{5})= & g(e_{6})=g(e_{7})=g(e_{8})=(\mathbb{R},\mathbb{R},0,0),
    \end{align*}
    and the reset maps are
    \begin{align*}
        r(e_{0})=&(x_{l},x_{u},\frac{z_{l}}{v_{z1}},\frac{z_{u}}{v_{z1}})\\
        r(e_{1})=&(\frac{x_{l}}{v_{x1}},\frac{x_{u}}{v_{x1}},z_{l},\frac{z_{u}}{v_{z2}}) \\
        r(e_{2})=&(\frac{x_{l}}{v_{x2}},\frac{x_{u}}{v_{x2}},z_{l},\frac{z_{u}}{v_{z2}}) \\
        r(e_{3})=&r(e_{4})=(x_{l},x_{u},-\frac{z_{l}}{v_{z1}},-\frac{z_{u}}{v_{z1}})\\
        r(e_{5})=&r(e_{6})=r(e_{7})=r(e_{8})=(0,0,0,0).
    \end{align*}
    \item So the set of locations is
    \begin{align*}
        L=\{&(\ell_{0},(0,0,0,0)),(\ell_{1},(0,0,\bot,\bot)),(\ell_{2},(\bot,\bot,\frac{h}{v_{z1}},\bot)),\\
        &(\ell_{3},(\bot,\bot,\frac{h}{v_{z1}},\bot)),(\ell_{4},(\frac{l}{v_{x1}},\frac{l}{v_{x1}},\bot,\bot)),(\ell_{4},(\frac{l}{v_{x2}},\frac{l}{v_{x2}},\bot,\bot)),\\
        &(\ell_{5},(0,0,0,0))\}
    \end{align*}
    \item The invariant conditions for each location are computed to be
    \begin{align*}
            \Inv(\ell_{0},(0,0,0,0))&=\Inv(\ell_{5},(0,0,0,0))=\emptyset \\
            \Inv(\ell_{1},(0,0,\bot,\bot))&=(\mathbb{R}_{\geq0},\mathbb{R}_{\geq0},[0,\infty),[0,\frac{h}{v_{z1}}]) \\
            \Inv(\ell_{2},(\bot,\bot,\frac{h}{v_{z1}},\bot))&=([0,\infty),[0,\frac{l}{v_{x1}}],\mathbb{R}_{\geq0},[0,\frac{h+\varepsilon}{v_{z2}}]) \\
            \Inv(\ell_{3},(\bot,\bot,\frac{h}{v_{z1}},\bot))&=([0,\infty),[0,\frac{l}{v_{x2}}],\mathbb{R}_{\geq0},[0,\frac{h+\varepsilon}{v_{z2}}]) \\
            \Inv(\ell_{4},(\frac{l}{v_{x1}},\frac{l}{v_{x1}},\bot,\bot))&=(\mathbb{R}_{\geq0},\mathbb{R}_{\geq0},[-\frac{h+\varepsilon}{v_{z1}},0],[-\frac{h+\varepsilon}{v_{z1}},0]) \\
            \Inv(\ell_{4},(\frac{l}{v_{x2}},\frac{l}{v_{x2}},\bot,\bot))&=(\mathbb{R}_{\geq0},\mathbb{R}_{\geq0},[-\frac{h+\varepsilon}{v_{z1}},0],[-\frac{h+\varepsilon}{v_{z1}},0])
    \end{align*}
    \item As this resulting automaton is a probabilistic timed automaton, the flow condition in each location, for all variables is $1$.
    \item The probabilistic transition function for each edge is computed to be
    \begin{align*}
        t((\ell_{0},(0,0,0,0)),\text{TakeOff})&=\{1:(\ell_{1},(0,0,\bot,\bot))\} \\
        t((\ell_{1},(0,0,\bot,\bot)),\text{Move})&=\{0.6:(\ell_{2},(\bot,\bot,\frac{h}{v_{z1}},\bot)),0.4:(\ell_{3},(\bot,\bot,\frac{h}{v_{z1}},\bot))\}\\
        t((\ell_{2},(\bot,\bot,\frac{h}{v_{z1}},\bot)),\text{Stop})&=\{1:(\ell_{4},(\frac{l}{v_{x1}},\frac{l}{v_{x1}},\bot,\bot))\} \\
        t((\ell_{3},(\bot,\bot,\frac{h}{v_{z1}},\bot)),\text{Stop})&=\{1:(\ell_{4},(\frac{l}{v_{x2}},\frac{l}{v_{x2}},\bot,\bot))\} \\
        t((\ell_{4},(\frac{l}{v_{x1}},\frac{l}{v_{x1}},\bot,\bot)),\text{Land})&=\{0.99:(\ell_{0},(0,0,0,0)),0.01:(\ell_{5},(0,0,0,0))\} \\
        t((\ell_{4},(\frac{l}{v_{x2}},\frac{l}{v_{x2}},\bot,\bot)),\text{Land})&=\{0.99:(\ell_{0},(0,0,0,0)),0.01:(\ell_{5},(0,0,0,0))\}.
    \end{align*}
\end{itemize}
\end{ex}


\subsection{Abstracting Stochastic Hybrid Automata}
We are now going to look into how a hybrid automaton with stochastic distributions can be abstracted into a probabilistic hybrid automaton. Should the output PHA have the right form we can translate it to a PTA (Section~\ref{sec:phatrans}) and verify the PTA.
Note that we use the notation of SHAs introduced in Section~\ref{sec:sha}.

\begin{defi}[Command Abstraction \cite{Hahn2012}]
Let $c=(g\rightarrow L)$ be a measurable continuous guarded command, choose $p_{i}\geq0$ with $1\leq i\leq n$ and $\sum_{i=1}^{n} p_{i} = 1$. Let
$\hat{u}_{1},\ldots,\hat{u}_{n}:Q\rightarrow \Sigma_{Q}$ and $u_{1},\ldots,u_{n} : Q\rightarrow \Sigma_{Q}$ be functions where $\forall q\in Q$ we have
\begin{itemize}
    \item $\hat{u}_{i}$ and $u_{i}$ are $\Sigma_{Q}-H(\Sigma_{Q})$-measurable $\forall i\in\{1,\ldots,n\}$;
    \item $L(q)(\hat{u}_{i}(q)) = p_{i}$;
    \item $L(q)(\bigcup_{i=1}^{n}\hat{u}_{i}(q)) = 1$;
    \item $\hat{u}_{1}(q),\ldots,\hat{u}_{n}(q)$ are pairwise disjoint;
    \item $\hat{u}_{i}(q)\subseteq u_{i}(q)$ $\forall i\in\{1,\ldots,\}$.
\end{itemize}
A \emph{command abstraction} is $\mathbf{f}=(\hat{u}_{1},\ldots,\hat{u}_{n},u_{1},\ldots,u_{n},p_{1},\ldots,p_{n})$ and the measurable finite guarded command is defined as $abs(c,\mathbf{f}) = (g\rightarrow p_{1}:u_{1}+\cdots+p_{n}:u_{n})$.
\end{defi}

To abstract the SHA into a LTS we first abstract it into a PHA, before abstracting that PHA to an PTA using the translations described in Section~\ref{sec:phatrans}. Note that the PHA here has (on first sight) different components to our earlier definition, this is just because some of the components are hidden within the definitions, or within other component, for example $\Inv$ is contained within the definition of the flow function $f$. Extracting these components is trivial, and shown in the example below.

\begin{defi}[SHA to PHA Abstraction \cite{Hahn2012}]
Let $\mathcal{S}=(V,L,\Init,f,\tfin,\tcont)$ be a SHA and $\mathbf{F}=\langle\mathbf{f}_{c}\rangle_{c\in \tcont}$ a family of command abstractions. Then the \emph{PHA abstraction of $\mathcal{S}$} is $abs(\mathcal{S},\mathbf{F})=(V,L,\Init,f,t)$ defined as follows
\begin{itemize}
    \item $V$, $L$, $\Init$ and $f$ are inherited from $\mathcal{S}$;
    \item $t$ is defined as the disjoint union of $\tfin\cup\{abs(c,\mathbf{f}_{c}) : c \in \tcont\}$.
\end{itemize}
\end{defi}


% \begin{defi}[Abstract State Space]
% An \emph{abstract state space} of a state space $Q=L\times \val(V)$ is a finite set $Q_{A} = \{ q_{1},\ldots,q_{n}\}\subseteq L\times 2^{\val(V)}$ where $q=(\ell,\theta)\in Q_{A}$ and we have $\bigcup_{(\ell,\theta)\in Q} \theta = \val(V)$ for all $\ell\in L$.
% \end{defi}
%
% \begin{rem}
%     $Q_{A}$ does not necessary need to be a partitioning of $L\times\val(V)$.
% \end{rem}
%
% \begin{defi}[Time Restriction]
% A \emph{time restriction} $\mathbf{T}=\langle \mathbf{t}_{\ell}\rangle_{\ell\in L}$ for $f$  is defined such that for each $\ell\in L$ we have
% \begin{itemize}
%     \item $\mathbf{t}_{\ell} : \val(V)\rightarrow \mathbb{R}_{\geq0}$
%     \item for each $v\in\val(V)$, $\mathbf{t}\geq0$ and $v'\in f(\ell)(v)(\mathbf{t})$ there is $n\geq1$ for which there are $v_{1}\ldots,v_{n}\in\val(V)$ and
%     $\mathbf{t}_{1},\ldots,\mathbf{t}_{n-1}\in\mathbb{R}_{\geq0}$ where
%         \begin{itemize}
%             \item $v=v_{1}$
%             \item $v'=v_{n}$
%             \item $\sum_{i} \mathbf{t}_{i} = \mathbf{t}$
%             \item for $i$ with $1\leq i\leq n$ we have $\mathbf{t}_{i}\leq \mathbf{t}_{\ell}(v_{i})$ and $v_{i+1}\in \Inv(\ell)$.
%         \end{itemize}
% \end{itemize}
% \end{defi}
%
% \begin{defi}[Lifted Distribution]
% Let $Q_{A}$ be an abstract state space and $p=[q_{1}\mapsto p_{1},\ldots,q_{n}\mapsto p_{n}]\in\mu(Q)$. The set of \emph{lifted distributions} is defined as
% \[
% Lift_{Q_{A}}(p) = \{[a_{1}\mapsto p_{1},\ldots,a_{n}\mapsto p_{n} : a_{1},\ldots,a_{n}\in Q_{A} \text{ and } q_{1}\in a_{1},\ldots, q_{n}\in a_{n}\}.
% \]
% \end{defi}
%
% \begin{rem}
% If $Q_{A}$ is a partitioning of $L\times \val(V)$ then $Lift_{Q_{A}}$ is a singleton.
% \end{rem}
%
% \begin{defi}[PHA to LTS Abstraction \cite{Hahn2012}]
% \label{def:phs2lts}
% Let $abs(\mathcal{S},\mathbf{F})=(V,L,\Init,f,t_{H})$ be a PHA abstraction of a SHA, $Q_{A}$ an abstract state space, and $\mathbf{T}$ a time restriction, then $\mathcal{P}=(Q,Q_{0},A,t_{P})$ is an \emph{abstraction} of $\mathcal{P}$ using $Q_{A}$ and $\mathbf{T}$ if
% \begin{itemize}
%     \item $Q=Q_{A}$
%     \item $(\Init,0,\ldots,0)\in Q_{0}$
%     \item $A=t_{H}\cup{\bot}$,
%     \item $\forall s\in Q_{A}$ and $q\in s$, $c=(g\rightarrow p_{1}:u_{1}+\cdots+p_{n}:u_{n})\in t_{H}$, if $q\in s\cap g$ then for all $p\in\{[s'_{1}\mapsto p_{1},\ldots,s'_{n}\mapsto p_{n}] : s'_{1}\in u_{1}(s),\ldots, s'_{n}\in u_{n}(s)\}$ there is $p_{A}\in Lift_{Q_{A}}(p)$ with $p_{A}\in t_{P}(s,c)$.
%     \item $\forall q\in Q_{A}$, $s=(\ell,v)\in q$, $\delta\in\mathbb{R}_{\geq 0}$ with $\delta\leq \mathbf{t}_{\ell}(v)$ and all $s'=(\ell,v')\in f(\ell)(s)(\delta)$ if $s'\notin q$ then there is $q'\in Q_{A}$ with $s'\in q'$ and $[q'\mapsto 1]\in t_{P}(q,\bot)$
% \end{itemize}
% \end{defi}
%
% \begin{rem}
% By $\bot$ we denote the single action that some time has passed.
% \end{rem}


\begin{ex}
For the stochastic hybrid automata we have to use a different example to illustrate the abstraction technique above.
In this example we have a UAV flying forward for a given distance, it might find an object on the way, if it does it "picks" the object up, and keeps flying until it reaches the final coordinates.
We check every two time steps whether something has been found or not.

\begin{figure}[H]
    \begin{center}
        \begin{tikzpicture}[node distance=5cm]%, initial text = {$x,y=0$}]
            \node [state,initial] (s0) {$\ell_{0}$\\$\dot{x}=v_{x},$\\$\dot{t}=1$\\$x\leq x_{max}$};
            \node [state] (s1) [right of = s0,node distance=5.8cm] {$\ell_{1}$\\ $\dot{x}=0,$\\ $\dot{t}=1$,\\ $t\leq1$};
            \node [state] (s2) [right of = s1,node distance=4.2cm] {$\ell_{2}$\\ $\dot{x}=v_{x},$\\ $\dot{t}=1$\\ $x\leq x_{max}$};
            \node [state] (s3) [below of = s1,node distance=2.3cm] {$\ell_{3}$\\ $\dot{x}=0,$\\ $\dot{t}=0$};

            \path[->] (s0) edge [bend left = 10] node [above] {$x<max$,$t=2$,$t:=0$,$o:=\mathcal{N}(x,5)$} (s1)
                    edge node [left] {$x=max$} (s3)
                (s1) edge  [bend left = 10] node [below] {$t\geq 1$,$o< 5$, $t:=0$} (s0)
                    edge node [above] {$t\geq 1, o\geq 5$, $t:=0$} (s2)
                (s2) edge node [right] {$x=max$} (s3);

        \end{tikzpicture}
        \caption{Example of UAV find stochastic hybrid automaton.}
         \label{fig:shauav}
     \end{center}
\end{figure}
We first abstract the continuous distribution functions into discrete distributions. There is only one transition with a continuous distribution in our example (with normal distribution) and the transition lies between location $\ell_{0}$ and $\ell_{1}$. Then for the command abstraction we let $c=(x<max\land t=2\rightarrow \mathcal{N}(x,5):\ell_{1})$. We choose $p_{1}=p_{2}=p_{3}=p_{4}=0.25$. As we have a normal distribution we can use symbolic approximation to find intervals $a_{i}$ which correlate to the probabilities $p_{i}$
\begin{align*}
    a_{1} &\in x+[-3.38,-3.37] \\
    a_{2} &\in x+[-0.1,0.1] \\
    a_{3} &\in x+[3.37,3.38] .
\end{align*}
This approximation changes for different distributions.
From these intervals we can find $\hat{u}_{i}$, for $q=(\ell_{0},x,t,o)$ we know that the reset will be $x:=x$ and $t:=0$.
\begin{align*}
    \hat{u}_{1}(q) & = \{(\ell_{1},x,0)\}\times(-\infty,a_{1}] \\
    \hat{u}_{2}(q) & = \{(\ell_{1},x,0)\}\times(a_{1},a_{2}] \\
    \hat{u}_{3}(q) & = \{(\ell_{1},x,0)\}\times(a_{2},a_{3}] \\
    \hat{u}_{4}(q) & = \{(\ell_{1},x,0)\}\times(a_{3},\infty).
\end{align*}
This sets $L(q)(u_{i}(q))=p_{i}=0.25$ for $i\in\{1,\ldots,4\}$ and $L(q)(\bigcup_{i=1}^{4}\hat{u}_{i}(q))=1$.
Then we can find the $u_{i}$ to be
\begin{align*}
    u_{1}(q) & = \{(\ell_{1},x,0)\}\times(-\infty,-3.37] \\
    u_{2}(q) & = \{(\ell_{1},x,0)\}\times[-3.38,0.1] \\
    u_{3}(q) & = \{(\ell_{1},x,0)\}\times[-0.1,3.38] \\
    u_{4}(q) & = \{(\ell_{1},x,0)\}\times[3.37,\infty) \\
\end{align*}
where $\hat{u}_{i}\subseteq u_{i}$. So we find that the command abstraction is $\mathbf{f}=(\hat{u}_{1},\ldots,\hat{u}_{4},u_{1},\ldots,u_{4},p_{1},\ldots,p_{4})$ and the abstracted guard is
\[
abs(c,\mathbf{f}) = (x<max\land t=2\rightarrow 0.25:u_{1}+\cdots+0.25:u_{4}).
\]

So the PHA that is the abstraction of the SHA in Figure~\ref{fig:shauav} consists of
\begin{itemize}
    \item $V=\{x,t,o\}$
    \item $L=\{\ell_{0},\ell_{1},\ell_{2},\ell_{3}\}$
    \item $\Init=\ell_{0}$
    \item the flow function is
        \begin{align*}
            f(\ell_{0}) &= \{\dot{x}=v_{x},\dot{t}=1,\dot{o}=0\} \\
            f(\ell_{1}) &= \{\dot{x}=0,\dot{t}=1,\dot{o}=0\} \\
            f(\ell_{2}) &= \{\dot{x}=v_{x},\dot{t}=1,\dot{o}=0\} \\
            f(\ell_{3}) &= \{\dot{x}=0,\dot{t}=0,\dot{o}=0\}
        \end{align*}
    \item the discrete transition function $t$ is
        \begin{align*}
            t(\ell_{0}) &= \{x=x_{max}\rightarrow 1:\ell_{3},\ x<max\land t=2\rightarrow 0.25:u_{1}+\cdots+0.25:u_{4}\} \\
            t(\ell_{1}) &= \{o\leq10\rightarrow 1:\ell_{0},\ o>10\rightarrow 1:\ell_{2}\} \\
            t(\ell_{2}) &= \{x=x_{max}\rightarrow 1:\ell_{3}\}.
        \end{align*}
\end{itemize}
Next we need to turn this PHA into a form that we can work with.

%There are two ways this can be approached. T We can reformulate the components to fit the definition of a rectangular PHA and then use the translations discussed in Section~\ref{sec:phatrans} to create a PTA which can be model checked. This rectangular PHA would be defined as
\begin{itemize}
    \item $V=\{x,t,o\}$
    \item $L=\{\ell_{0},\ell_{1,1},\ell_{1,2},\ell_{1,3},\ell_{1,4},\ell_{2},\ell_{3}\}$
    \item $A=\{act_{1},act_{2},act_{3},act_{4},act_{5}\}$, we can choose our action set freely.
    \item $\Init = \{\ell_{0},(\mathbb{R},0,0)\}$
    \item The invariant conditions are
        \begin{align*}
            \Inv(\ell_{0})&=((-\infty,x_{max}],[0,\infty),\mathbb{R}) \\
            \Inv(\ell_{1,1})&=\Inv(\ell_{1,2})=\Inv(\ell_{1,3})=\Inv(\ell_{1,4})=(\mathbb{R},[0,1],\mathbb{R}) \\
            \Inv(\ell_{2})&=((-\infty,x_{max}],[0,\infty),\mathbb{R}) \\
            \Inv(\ell_{3})&=\emptyset
        \end{align*}
    \item The edge set is
    \begin{align*}
        E=\{&e_{1,1}=(\ell_{0},act_{1},\ell_{1,1}),
        e_{1,2}=(\ell_{0},act_{1},\ell_{1,2}), e_{1,3}=(\ell_{0},act_{1},\ell_{1,3}),
        e_{1,4}=(\ell_{0},act_{1},\ell_{1,4}), \\
        & e_{2}=(\ell_{0},act_{2},\ell_{3}), e_{3,1}=(\ell_{1,1},act_{3},\ell_{0}), e_{3,2}=(\ell_{1,2},act_{3},\ell_{0}), e_{3,3}=(\ell_{1,3},act_{3},\ell_{0}), \\
        & e_{3,4}=(\ell_{1,4},act_{3},\ell_{0}), e_{4,1}=(\ell_{1,1},act_{4},\ell_{2}), e_{4,2}=(\ell_{1,2},act_{4},\ell_{2}), e_{4,3}=(\ell_{1,3},act_{4},\ell_{2}), \\
        & e_{4,4}=(\ell_{1,4},act_{4},\ell_{2}), e_{5}=(\ell_{2},act_{5},\ell_{3})\}
    \end{align*}
        with guard conditions
        \begin{align*}
            g(e_{1,1})=&g(e_{1,2})=g(e_{1,3})=g(e_{1,4})=((-\infty,x_{max}),[2,2],\mathbb{R}), \\
            g(e_{2})=&([x_{max},x_{max}],\mathbb{R},\mathbb{R}), \\
            g(e_{3,1})=&g(e_{3,2})=g(e_{3,3})=g(e_{3,4})=(\mathbb{R},[1,\infty),(-\infty,5]), \\
            g(e_{4,1})=&g(e_{4,2})=g(e_{4,4})=g(e_{4,4})=(\mathbb{R},[1,\infty),[5,\infty)), \\
            g(e_{5})=&([x_{max},x_{max}],\mathbb{R},\mathbb{R})
        \end{align*}
        and reset maps
        \begin{align*}
            r(e_{1,1})=&(x,0,(-\infty,-3.37]) \\
            r(e_{1,2})=&(x,0,[-3.38,0.1]) \\
            r(e_{1,3})=&(x,0,[-0.1,3.38]) \\
            r(e_{1,4})=&(x,0,[3.37,\infty)) \\
            r(e_{2})=&(x,t,o) \\
            r(e_{3,1})=&r(e_{3,2})=r(e_{3,3})=r(e_{3,4})=(x,0,o) \\
            r(e_{4,1})=&r(e_{4,2})=r(e_{4,3})=r(e_{4,4})=(x,0,o) \\
            r(e_{5})=&(x,t,o).
        \end{align*}
    \item The flow conditions for each location are
        \begin{align*}
            f(\ell_{0})=&\{\dot{x}=v_{x},\dot{t}=1,\dot{o}=0\} \\
            f(\ell_{1,1})=&f(\ell_{1,2})=f(\ell_{1,3})=f(\ell_{1,4})=\{\dot{x}=0,\dot{t}=1,\dot{o}=0\}\\
            f(\ell_{2})=&\{\dot{x}=v_{x},\dot{t}=1,\dot{o}=0\} \\
            f(\ell_{3})&=\{\dot{x}=0,\dot{t}=0,\dot{o}=0\}.
        \end{align*}
    \item The probabilistic transition function is
        \begin{align*}
            t(\ell_{0},act_{1})=&\{0.25:\ell_{1,1},0.25:\ell_{1,2},0.25:\ell_{1,3},0.25:\ell_{1,4}\} \\
            t(\ell_{0},act_{2})=&\{1:\ell_{3}\} \\
            t(\ell_{1,1},act_{3})=&t(\ell_{1,2},act_{3})=t(\ell_{1,4},act_{3})=t(\ell_{1,4},act_{3})=\{1:\ell_{0}\} \\
            t(\ell_{1,1},act_{4})=&t(\ell_{1,2},act_{4})=t(\ell_{1,4},act_{4})=t(\ell_{1,4},act_{4})=\{1:\ell_{2}\} \\
            t(\ell_{2},act_{5})=&\{1:\ell_{3}\}.
        \end{align*}
\end{itemize}
The above automaton is clearly a rectangular probabilistic hybrid automaton, which we can translate into a PTA.
% Instead we will use the abstraction technique described earlier in this section in Definition~\ref{def:phs2lts}.
% We choose the abstract state space to be
% \begin{align*}
% Q_{A}=\{&(\ell_{0},(-\infty,x_{max}),(-\infty,2],\mathbb{R}), (\ell_{0},[x_{max},\infty),(2,\infty),\mathbb{R}), (\ell_{1},\mathbb{R},(-\infty,1],(-\infty,-3.37]), \\
% & (\ell_{1},\mathbb{R},(-\infty,1],[-3.38,0.1]), (\ell_{1},\mathbb{R},(-\infty,1],[-0.1,3.38]), (\ell_{1},\mathbb{R},(-\infty,1],[3.37,\infty)) \\
% & (\ell_{1},\mathbb{R},(1,\infty),(-\infty,-3.37]), (\ell_{1},\mathbb{R},(1,\infty),[-3.38,0.1]), (\ell_{1},\mathbb{R},(1,\infty),[-0.1,3.38]), \\
% & (\ell_{1},\mathbb{R},(1,\infty),[3.37,\infty)), (\ell_{2},(-\infty,x_{max}),\mathbb{R},\mathbb{R}), (\ell_{2},[x_{max},\infty),\mathbb{R},\mathbb{R}), \\
% & (\ell_{3},\mathbb{R},\mathbb{R},\mathbb{R})
% \}
% \end{align*}
% by using the the invariance and guard conditions of the locations. Similarly we define the time restrictions to be $\mathbf{T}=\{\mathbf{t}_{\ell{0}},\mathbf{t}_{\ell{1}},\mathbf{t}_{\ell{2}},\mathbf{t}_{\ell{3}}\}$
% \begin{align*}
%     \mathbf{t}_{\ell{0}}=& \frac{x_{max}}{v_{x}}\\
%     \mathbf{t}_{\ell{1}}=& 1\\
%     \mathbf{t}_{\ell{2}}=& \frac{x_{max}}{v_{x}}\\
%     \mathbf{t}_{\ell{3}}=& 10.
% \end{align*}
% based on the continuous behaviour of the variables, the guard and invariance conditions. As there is no continuous flow in $\ell_{3}$ we set an arbitrary time out.
%
% With $Q_{A}$ and $\mathbb{T}$ we can now find the LTS abstraction of our PHA, to be
% \begin{itemize}
%     \item $Q=Q_{A}$
%     \item $Q_{0}=\{(\ell_{0},0,0,0)\}$
%     \item $A=\{t(\ell_{0}),t(\ell_{1}),t(\ell_{2}),\bot\}$
%     \item
% \end{itemize}
\end{ex}


\subsection{Abstracting Discrete Time Stochastic Hybrid Systems}
In this section we will show how to abstract a DTSHS $\mathcal{D}=(L,n,\Init,T_x,T_q,R)$ to a discrete time markov chain, which then can be model checked using the usual techniques.

Before we get to the abstraction we give a couple preliminary definitions and notations.

\begin{defi}[Partition of the State Space]
Let $\mathcal{Q}=\bigcup_{\ell\in L}\{\ell\} \times Q_{\ell}$ be the state space, where $Q_{\ell}\subset \mathbb{R}^{n(\ell)}$ (see the Definition~\ref{def:DTSHS} for $n$) is a compact
set, with $Q_{\ell} = \bigcup_{i=1}^{m_{\ell}} Q_{\ell,i}$ having a finite partition and $Q_{\ell,i}\in \mathcal{B}(\mathbb{R}^{n(\ell)})$ $Q_{\ell,i}\cap Q_{\ell,j} = \emptyset$ for all
$i\neq j$, $m_{\ell}\in\mathbb{N}$ is number of partitions of the continous domain of $\ell$.
\end{defi}

\begin{defi}[Partition Representative]
We define $r_{\ell}:\mathcal{B}(\mathbb{R}^{n(\ell)})\rightarrow\mathbb{R}^{n(\ell)}$ to be the function returning the \emph{representative point} $r_{\ell}(Q_{\ell,i})$ of the partition $Q_{\ell,i}$, which is a randomly chosen point given
 $Q_{\ell,i}\in\mathcal{B}(\mathbb{R}^{n(\ell)})$ and $\ell\in L$.
\end{defi}


\begin{defi}[DTSHS to DTMC Abstraction \cite{Abate2010,Abate2011}]
Let $\mathcal{D}=(L,n,\Init,T_x,T_q,R)$ be a DTSHS. Then the DTMC associated to $\mathcal{D}$ $\mathcal{C}=(\mathcal{Q}_{\delta},\Init_{\delta},t_{\delta})$ is defined as
\begin{itemize}
    \item $\mathcal{Q}_{\delta}=\{(\ell,i) : \ell \in L, i\in\{1,\ldots,m_{\ell}\}\}$ is the state space, $m_{\ell}\in \mathbb{N}$;
    \item $\Init_{\delta}(\ell,i) = \int_{Q_{\ell,i}} \Init(\ell,x) dx$ is the probability distribution for the initial state;
    \item $t_{\delta}((\ell,i),(\ell',i')) = T(\ell'\times Q_{\ell',i'}, (\ell,r_{\ell}(Q_{\ell,i})))$ is the transition probability matrix.
\end{itemize}
\end{defi}

%\begin{defi}[Gridding]
%Let $\delta_{\ell,i} = \sup\{ \| x-x'\| : x,x'\in Q_{\ell,i}\}$ be the \emph{diameter} of $Q_{\ell,i}$. We also define $\delta=\max_{i\in\{1,\ldots,m_{\ell},\ell\in L} \delta_{\ell,i}$ to be the \emph{grid size parameter},
%\end{defi}


%\cite{Abate2010,Amin2006,Hahn2011,Hofbaur2002,Julius2009,Kattenbelt2009,Soudjani2011a,Bujorianu2004,Koutsoukos2006,Hu2000,Prandini2006}

%\cite{Hahn2011} uses game-theory to produce the abstraction and an optimal choice.

%\cite{Kattenbelt2009} uses a mix of predicate and CEGAR.
