A hybrid system is a system where both continuous and discrete events can occur. In the real world there numerous examples of such systems, for example mutual-exclusion protocols \cite{Alur1995a}, temperature regulation \cite{Nicollin1993,Amin2006,Raskin2005}, water level regulation \cite{Alur1993}, train controllers \cite{Platzer2011a}, aircraft landing \cite{Tomlin2003}, robot cooperation \cite{Chaimowicz2003} and robotic decision making \cite{Dennis2013a}.

Being able to verify a hybrid system is crucial to show that the modeled real world application is safe \cite{Livadas1998a,Prajna2007}. There are various techniques on how to build a verifiable representation of hybrid systems, as well as numerous tools. This paper focuses on abstraction techniques, which yield a finite system, or a system that can be verified using current model checking approaches.

This survey is structured as follows, we will introduce hybrid systems and their representation formally in Section~\ref{sec:intro}. In the same section we will also discuss the different types of hybrid systems. In the following sections we will then present the different types of abstractions of hybrid systems, with results and tools which utilise the techniques. Finally, we will look at some related related work, which might influence further abstraction techniques or the advance of the current techniques.
