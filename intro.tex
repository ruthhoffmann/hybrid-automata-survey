We model and analyse complex systems using a range of mathematical formalisms and techniques. The type of system dictates the formalism used. For example, a system exhibiting continuous behaviour (a system of chemical reactions, fluid flow, movement of an aircraft, say) can be modelled using a set of differential equations, whereas a discrete state system can be modelled as finite state transition systems (or Finite State Automata---FSAs), or as Markov chain variants. An extension to finite state automata including real-timed clocks---Timed Automata (TAs) are used to model systems with additional time-dependent behaviour, such as real time systems or networks.


Some systems exhibit both discrete and continuous behaviour and so can not be modelled as a FSA or a set of differential equations alone (for example mutual-exclusion protocols \cite{Alur1995a}, temperature regulation \cite{Nicollin1993,Amin2006,Raskin2005}, water level regulation \cite{Alur1993}, train controllers \cite{Platzer2011a}, aircraft landing \cite{Tomlin2003}, robot cooperation \cite{Chaimowicz2003} and robotic decision making \cite{Dennis2013}).
These systems are known as hybrid systems and can be modelled using hybrid automata (HAs)---where continuous behaviour is modelled using differential equations within a finite set of {\it locations} (or {\it modes}) and discrete behaviour as discrete transitions between the modes. Being able to verify a hybrid system is crucial to show that the modelled real world application is safe \cite{Livadas1998a,Prajna2007}. However, unlike FSAs, which can (at least theoretically) be analysed using explicit or symbolic search of the statespace, HAs --- thanks to their continuous component---can not (although the state reachability problem for TAs, which are a special class of HAs is decidable \cite{Alur1994}).
There are various techniques on how to build a verifiable representation of a HA, as well as numerous tools. This paper focuses on abstraction techniques, which yield a finite system, or a system that can be verified using current model checking approaches. The abstractions do not use common notation, and some are special cases. We unify these.

Other systems, in addition to continuous and discrete behaviour, have probabilistic (stochastic) attributes, and can be modelled using stochastic hybrid automata (SHAs). When this stochastic behaviour translates as probabilistic transitions between locations, the automata are known as Probabilistic Hybrid Automata (PHAs). Otherwise, the stochastic behaviour is within the locations themselves (and represented as a set of distributions).
Like HAs, the verification of SHAs requires abstraction to a simpler, verifiable form, often using techniques similar to those used for their non-stochastic counterparts.

In this paper, we survey HAs and SHAs and their abstractions. We first introduce HAs and SHAs (Sections~\ref{sec:hybrid} and \ref{sec:stoch} respectively). In Section~\ref{sec:background} we provide some additional relevant background material and in
Sections~\ref{sec:abs} and \ref{sec:propabs} we present the different abstraction techniques of HAs and PHAs respectively, and a translation of SHAs to PHAs. %Finally, in Section~\ref{sec:related} we present relevant additional related work and tools. \comm{May not need last section?}
