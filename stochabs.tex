We are now going to look into how a hybrid automaton with stochastic distributions can be abstracted into a probabilistic hybrid automaton. Should the output PHA have the right form we can translate it to a PTA (Section~\ref{sec:phatrans}) and verify the PTA.
Note that we use the notation of SHAs introduced in Section~\ref{sec:sha}.

\begin{defi}[Command Abstraction \cite{Hahn2012}]
Let $c=(g\rightarrow L)$ be a measurable continuous guarded command, choose $p_{i}\geq0$ with $1\leq i\leq n$ and $\sum_{i=1}^{n} p_{i} = 1$. Let
$\hat{u}_{1},\ldots,\hat{u}_{n}:Q\rightarrow \Sigma_{Q}$ and $u_{1},\ldots,u_{n} : Q\rightarrow \Sigma_{Q}$ be functions where $\forall q\in Q$ we have
\begin{itemize}
    \item $\hat{u}_{i}$ and $u_{i}$ are $\Sigma_{Q}-H(\Sigma_{Q})$-measurable $\forall i\in\{1,\ldots,n\}$;
    \item $L(q)(\hat{u}_{i}(q)) = p_{i}$;
    \item $L(q)(\bigcup_{i=1}^{n}\hat{u}_{i}(q)) = 1$;
    \item $\hat{u}_{1}(q),\ldots,\hat{u}_{n}(q)$ are pairwise disjoint;
    \item $\hat{u}_{i}(q)\subseteq u_{i}(q)$ $\forall i\in\{1,\ldots,\}$.
\end{itemize}
A \emph{command abstraction} is $\mathbf{f}=(\hat{u}_{1},\ldots,\hat{u}_{n},u_{1},\ldots,u_{n},p_{1},\ldots,p_{n})$ and the measurable finite guarded command is defined as $abs(c,\mathbf{f}) = (g\rightarrow p_{1}:u_{1}+\cdots+p_{n}:u_{n})$.
\end{defi}

To abstract the SHA into a LTS we first abstract it into a PHA, before abstracting that PHA to an PTA using the translations described in Section~\ref{sec:phatrans}. Note that the PHA here has (on first sight) different components to our earlier definition, this is just because some of the components are hidden within the definitions, or within other component, for example $\Inv$ is contained within the definition of the flow function $f$. Extracting these components is trivial, and shown in the example below.

\begin{defi}[SHA to PHA Abstraction \cite{Hahn2012}]
Let $\mathcal{S}=(V,L,\Init,f,\tfin,\tcont)$ be a SHA and $\mathbf{F}=\langle\mathbf{f}_{c}\rangle_{c\in \tcont}$ a family of command abstractions. Then the \emph{PHA abstraction of $\mathcal{S}$} is $abs(\mathcal{S},\mathbf{F})=(V,L,\Init,f,t)$ defined as follows
\begin{itemize}
    \item $V$, $L$, $\Init$ and $f$ are inherited from $\mathcal{S}$;
    \item $t$ is defined as the disjoint union of $\tfin\cup\{abs(c,\mathbf{f}_{c}) : c \in \tcont\}$.
\end{itemize}
\end{defi}


% \begin{defi}[Abstract State Space]
% An \emph{abstract state space} of a state space $Q=L\times \val(V)$ is a finite set $Q_{A} = \{ q_{1},\ldots,q_{n}\}\subseteq L\times 2^{\val(V)}$ where $q=(\ell,\theta)\in Q_{A}$ and we have $\bigcup_{(\ell,\theta)\in Q} \theta = \val(V)$ for all $\ell\in L$.
% \end{defi}
%
% \begin{rem}
%     $Q_{A}$ does not necessary need to be a partitioning of $L\times\val(V)$.
% \end{rem}
%
% \begin{defi}[Time Restriction]
% A \emph{time restriction} $\mathbf{T}=\langle \mathbf{t}_{\ell}\rangle_{\ell\in L}$ for $f$  is defined such that for each $\ell\in L$ we have
% \begin{itemize}
%     \item $\mathbf{t}_{\ell} : \val(V)\rightarrow \mathbb{R}_{\geq0}$
%     \item for each $v\in\val(V)$, $\mathbf{t}\geq0$ and $v'\in f(\ell)(v)(\mathbf{t})$ there is $n\geq1$ for which there are $v_{1}\ldots,v_{n}\in\val(V)$ and
%     $\mathbf{t}_{1},\ldots,\mathbf{t}_{n-1}\in\mathbb{R}_{\geq0}$ where
%         \begin{itemize}
%             \item $v=v_{1}$
%             \item $v'=v_{n}$
%             \item $\sum_{i} \mathbf{t}_{i} = \mathbf{t}$
%             \item for $i$ with $1\leq i\leq n$ we have $\mathbf{t}_{i}\leq \mathbf{t}_{\ell}(v_{i})$ and $v_{i+1}\in \Inv(\ell)$.
%         \end{itemize}
% \end{itemize}
% \end{defi}
%
% \begin{defi}[Lifted Distribution]
% Let $Q_{A}$ be an abstract state space and $p=[q_{1}\mapsto p_{1},\ldots,q_{n}\mapsto p_{n}]\in\mu(Q)$. The set of \emph{lifted distributions} is defined as
% \[
% Lift_{Q_{A}}(p) = \{[a_{1}\mapsto p_{1},\ldots,a_{n}\mapsto p_{n} : a_{1},\ldots,a_{n}\in Q_{A} \text{ and } q_{1}\in a_{1},\ldots, q_{n}\in a_{n}\}.
% \]
% \end{defi}
%
% \begin{rem}
% If $Q_{A}$ is a partitioning of $L\times \val(V)$ then $Lift_{Q_{A}}$ is a singleton.
% \end{rem}
%
% \begin{defi}[PHA to LTS Abstraction \cite{Hahn2012}]
% \label{def:phs2lts}
% Let $abs(\mathcal{S},\mathbf{F})=(V,L,\Init,f,t_{H})$ be a PHA abstraction of a SHA, $Q_{A}$ an abstract state space, and $\mathbf{T}$ a time restriction, then $\mathcal{P}=(Q,Q_{0},A,t_{P})$ is an \emph{abstraction} of $\mathcal{P}$ using $Q_{A}$ and $\mathbf{T}$ if
% \begin{itemize}
%     \item $Q=Q_{A}$
%     \item $(\Init,0,\ldots,0)\in Q_{0}$
%     \item $A=t_{H}\cup{\bot}$,
%     \item $\forall s\in Q_{A}$ and $q\in s$, $c=(g\rightarrow p_{1}:u_{1}+\cdots+p_{n}:u_{n})\in t_{H}$, if $q\in s\cap g$ then for all $p\in\{[s'_{1}\mapsto p_{1},\ldots,s'_{n}\mapsto p_{n}] : s'_{1}\in u_{1}(s),\ldots, s'_{n}\in u_{n}(s)\}$ there is $p_{A}\in Lift_{Q_{A}}(p)$ with $p_{A}\in t_{P}(s,c)$.
%     \item $\forall q\in Q_{A}$, $s=(\ell,v)\in q$, $\delta\in\mathbb{R}_{\geq 0}$ with $\delta\leq \mathbf{t}_{\ell}(v)$ and all $s'=(\ell,v')\in f(\ell)(s)(\delta)$ if $s'\notin q$ then there is $q'\in Q_{A}$ with $s'\in q'$ and $[q'\mapsto 1]\in t_{P}(q,\bot)$
% \end{itemize}
% \end{defi}
%
% \begin{rem}
% By $\bot$ we denote the single action that some time has passed.
% \end{rem}


\begin{ex}
For the stochastic hybrid automata we have to use a different example to illustrate the abstraction technique above.
In this example we have a UAV flying forward for a given distance, it might find an object on the way, if it does it "picks" the object up, and keeps flying until it reaches the final coordinates.
We check every two time steps whether something has been found or not.

\begin{figure}[H]
    \begin{center}
        \begin{tikzpicture}[node distance=5cm]%, initial text = {$x,y=0$}]
            \node [state,initial] (s0) {$\ell_{0}$\\$\dot{x}=v_{x},$\\$\dot{t}=1$\\$x\leq x_{max}$};
            \node [state] (s1) [right of = s0,node distance=5.8cm] {$\ell_{1}$\\ $\dot{x}=0,$\\ $\dot{t}=1$,\\ $t\leq1$};
            \node [state] (s2) [right of = s1,node distance=4.2cm] {$\ell_{2}$\\ $\dot{x}=v_{x},$\\ $\dot{t}=1$\\ $x\leq x_{max}$};
            \node [state] (s3) [below of = s1,node distance=2.3cm] {$\ell_{3}$\\ $\dot{x}=0,$\\ $\dot{t}=0$};

            \path[->] (s0) edge [bend left = 10] node [above] {$x<max$,$t=2$,$t:=0$,$o:=\mathcal{N}(x,5)$} (s1)
                    edge node [left] {$x=max$} (s3)
                (s1) edge  [bend left = 10] node [below] {$t\geq 1$,$o< 5$, $t:=0$} (s0)
                    edge node [above] {$t\geq 1, o\geq 5$, $t:=0$} (s2)
                (s2) edge node [right] {$x=max$} (s3);

        \end{tikzpicture}
        \caption{Example of UAV find stochastic hybrid automaton.}
         \label{fig:shauav}
     \end{center}
\end{figure}
We first abstract the continuous distribution functions into discrete distributions. There is only one transition with a continuous distribution in our example (with normal distribution) and the transition lies between location $\ell_{0}$ and $\ell_{1}$. Then for the command abstraction we let $c=(x<max\land t=2\rightarrow \mathcal{N}(x,5):\ell_{1})$. We choose $p_{1}=p_{2}=p_{3}=p_{4}=0.25$. As we have a normal distribution we can use symbolic approximation to find intervals $a_{i}$ which correlate to the probabilities $p_{i}$
\begin{align*}
    a_{1} &\in x+[-3.38,-3.37] \\
    a_{2} &\in x+[-0.1,0.1] \\
    a_{3} &\in x+[3.37,3.38] .
\end{align*}
This approximation changes for different distributions.
From these intervals we can find $\hat{u}_{i}$, for $q=(\ell_{0},x,t,o)$ we know that the reset will be $x:=x$ and $t:=0$.
\begin{align*}
    \hat{u}_{1}(q) & = \{(\ell_{1},x,0)\}\times(-\infty,a_{1}] \\
    \hat{u}_{2}(q) & = \{(\ell_{1},x,0)\}\times(a_{1},a_{2}] \\
    \hat{u}_{3}(q) & = \{(\ell_{1},x,0)\}\times(a_{2},a_{3}] \\
    \hat{u}_{4}(q) & = \{(\ell_{1},x,0)\}\times(a_{3},\infty).
\end{align*}
This sets $L(q)(u_{i}(q))=p_{i}=0.25$ for $i\in\{1,\ldots,4\}$ and $L(q)(\bigcup_{i=1}^{4}\hat{u}_{i}(q))=1$.
Then we can find the $u_{i}$ to be
\begin{align*}
    u_{1}(q) & = \{(\ell_{1},x,0)\}\times(-\infty,-3.37] \\
    u_{2}(q) & = \{(\ell_{1},x,0)\}\times[-3.38,0.1] \\
    u_{3}(q) & = \{(\ell_{1},x,0)\}\times[-0.1,3.38] \\
    u_{4}(q) & = \{(\ell_{1},x,0)\}\times[3.37,\infty) \\
\end{align*}
where $\hat{u}_{i}\subseteq u_{i}$. So we find that the command abstraction is $\mathbf{f}=(\hat{u}_{1},\ldots,\hat{u}_{4},u_{1},\ldots,u_{4},p_{1},\ldots,p_{4})$ and the abstracted guard is
\[
abs(c,\mathbf{f}) = (x<max\land t=2\rightarrow 0.25:u_{1}+\cdots+0.25:u_{4}).
\]

So the PHA that is the abstraction of the SHA in Figure~\ref{fig:shauav} consists of
\begin{itemize}
    \item $V=\{x,t,o\}$
    \item $L=\{\ell_{0},\ell_{1},\ell_{2},\ell_{3}\}$
    \item $\Init=\ell_{0}$
    \item the flow function is
        \begin{align*}
            f(\ell_{0}) &= \{\dot{x}=v_{x},\dot{t}=1,\dot{o}=0\} \\
            f(\ell_{1}) &= \{\dot{x}=0,\dot{t}=1,\dot{o}=0\} \\
            f(\ell_{2}) &= \{\dot{x}=v_{x},\dot{t}=1,\dot{o}=0\} \\
            f(\ell_{3}) &= \{\dot{x}=0,\dot{t}=0,\dot{o}=0\}
        \end{align*}
    \item the discrete transition function $t$ is
        \begin{align*}
            t(\ell_{0}) &= \{x=x_{max}\rightarrow 1:\ell_{3},\ x<max\land t=2\rightarrow 0.25:u_{1}+\cdots+0.25:u_{4}\} \\
            t(\ell_{1}) &= \{o\leq10\rightarrow 1:\ell_{0},\ o>10\rightarrow 1:\ell_{2}\} \\
            t(\ell_{2}) &= \{x=x_{max}\rightarrow 1:\ell_{3}\}.
        \end{align*}
\end{itemize}
Next we need to turn this PHA into a form that we can work with.

%There are two ways this can be approached. T We can reformulate the components to fit the definition of a rectangular PHA and then use the translations discussed in Section~\ref{sec:phatrans} to create a PTA which can be model checked. This rectangular PHA would be defined as
\begin{itemize}
    \item $V=\{x,t,o\}$
    \item $L=\{\ell_{0},\ell_{1,1},\ell_{1,2},\ell_{1,3},\ell_{1,4},\ell_{2},\ell_{3}\}$
    \item $A=\{act_{1},act_{2},act_{3},act_{4},act_{5}\}$, we can choose our action set freely.
    \item $\Init = \{\ell_{0},(\mathbb{R},0,0)\}$
    \item The invariant conditions are
        \begin{align*}
            \Inv(\ell_{0})&=((-\infty,x_{max}],[0,\infty),\mathbb{R}) \\
            \Inv(\ell_{1,1})&=\Inv(\ell_{1,2})=\Inv(\ell_{1,3})=\Inv(\ell_{1,4})=(\mathbb{R},[0,1],\mathbb{R}) \\
            \Inv(\ell_{2})&=((-\infty,x_{max}],[0,\infty),\mathbb{R}) \\
            \Inv(\ell_{3})&=\emptyset
        \end{align*}
    \item The edge set is
    \begin{align*}
        E=\{&e_{1,1}=(\ell_{0},act_{1},\ell_{1,1}),
        e_{1,2}=(\ell_{0},act_{1},\ell_{1,2}), e_{1,3}=(\ell_{0},act_{1},\ell_{1,3}),
        e_{1,4}=(\ell_{0},act_{1},\ell_{1,4}), \\
        & e_{2}=(\ell_{0},act_{2},\ell_{3}), e_{3,1}=(\ell_{1,1},act_{3},\ell_{0}), e_{3,2}=(\ell_{1,2},act_{3},\ell_{0}), e_{3,3}=(\ell_{1,3},act_{3},\ell_{0}), \\
        & e_{3,4}=(\ell_{1,4},act_{3},\ell_{0}), e_{4,1}=(\ell_{1,1},act_{4},\ell_{2}), e_{4,2}=(\ell_{1,2},act_{4},\ell_{2}), e_{4,3}=(\ell_{1,3},act_{4},\ell_{2}), \\
        & e_{4,4}=(\ell_{1,4},act_{4},\ell_{2}), e_{5}=(\ell_{2},act_{5},\ell_{3})\}
    \end{align*}
        with guard conditions
        \begin{align*}
            g(e_{1,1})=&g(e_{1,2})=g(e_{1,3})=g(e_{1,4})=((-\infty,x_{max}),[2,2],\mathbb{R}), \\
            g(e_{2})=&([x_{max},x_{max}],\mathbb{R},\mathbb{R}), \\
            g(e_{3,1})=&g(e_{3,2})=g(e_{3,3})=g(e_{3,4})=(\mathbb{R},[1,\infty),(-\infty,5]), \\
            g(e_{4,1})=&g(e_{4,2})=g(e_{4,4})=g(e_{4,4})=(\mathbb{R},[1,\infty),[5,\infty)), \\
            g(e_{5})=&([x_{max},x_{max}],\mathbb{R},\mathbb{R})
        \end{align*}
        and reset maps
        \begin{align*}
            r(e_{1,1})=&(x,0,(-\infty,-3.37]) \\
            r(e_{1,2})=&(x,0,[-3.38,0.1]) \\
            r(e_{1,3})=&(x,0,[-0.1,3.38]) \\
            r(e_{1,4})=&(x,0,[3.37,\infty)) \\
            r(e_{2})=&(x,t,o) \\
            r(e_{3,1})=&r(e_{3,2})=r(e_{3,3})=r(e_{3,4})=(x,0,o) \\
            r(e_{4,1})=&r(e_{4,2})=r(e_{4,3})=r(e_{4,4})=(x,0,o) \\
            r(e_{5})=&(x,t,o).
        \end{align*}
    \item The flow conditions for each location are
        \begin{align*}
            f(\ell_{0})=&\{\dot{x}=v_{x},\dot{t}=1,\dot{o}=0\} \\
            f(\ell_{1,1})=&f(\ell_{1,2})=f(\ell_{1,3})=f(\ell_{1,4})=\{\dot{x}=0,\dot{t}=1,\dot{o}=0\}\\
            f(\ell_{2})=&\{\dot{x}=v_{x},\dot{t}=1,\dot{o}=0\} \\
            f(\ell_{3})&=\{\dot{x}=0,\dot{t}=0,\dot{o}=0\}.
        \end{align*}
    \item The probabilistic transition function is
        \begin{align*}
            t(\ell_{0},act_{1})=&\{0.25:\ell_{1,1},0.25:\ell_{1,2},0.25:\ell_{1,3},0.25:\ell_{1,4}\} \\
            t(\ell_{0},act_{2})=&\{1:\ell_{3}\} \\
            t(\ell_{1,1},act_{3})=&t(\ell_{1,2},act_{3})=t(\ell_{1,4},act_{3})=t(\ell_{1,4},act_{3})=\{1:\ell_{0}\} \\
            t(\ell_{1,1},act_{4})=&t(\ell_{1,2},act_{4})=t(\ell_{1,4},act_{4})=t(\ell_{1,4},act_{4})=\{1:\ell_{2}\} \\
            t(\ell_{2},act_{5})=&\{1:\ell_{3}\}.
        \end{align*}
\end{itemize}
The above automaton is clearly a rectangular probabilistic hybrid automaton, which we can translate into a PTA.
% Instead we will use the abstraction technique described earlier in this section in Definition~\ref{def:phs2lts}.
% We choose the abstract state space to be
% \begin{align*}
% Q_{A}=\{&(\ell_{0},(-\infty,x_{max}),(-\infty,2],\mathbb{R}), (\ell_{0},[x_{max},\infty),(2,\infty),\mathbb{R}), (\ell_{1},\mathbb{R},(-\infty,1],(-\infty,-3.37]), \\
% & (\ell_{1},\mathbb{R},(-\infty,1],[-3.38,0.1]), (\ell_{1},\mathbb{R},(-\infty,1],[-0.1,3.38]), (\ell_{1},\mathbb{R},(-\infty,1],[3.37,\infty)) \\
% & (\ell_{1},\mathbb{R},(1,\infty),(-\infty,-3.37]), (\ell_{1},\mathbb{R},(1,\infty),[-3.38,0.1]), (\ell_{1},\mathbb{R},(1,\infty),[-0.1,3.38]), \\
% & (\ell_{1},\mathbb{R},(1,\infty),[3.37,\infty)), (\ell_{2},(-\infty,x_{max}),\mathbb{R},\mathbb{R}), (\ell_{2},[x_{max},\infty),\mathbb{R},\mathbb{R}), \\
% & (\ell_{3},\mathbb{R},\mathbb{R},\mathbb{R})
% \}
% \end{align*}
% by using the the invariance and guard conditions of the locations. Similarly we define the time restrictions to be $\mathbf{T}=\{\mathbf{t}_{\ell{0}},\mathbf{t}_{\ell{1}},\mathbf{t}_{\ell{2}},\mathbf{t}_{\ell{3}}\}$
% \begin{align*}
%     \mathbf{t}_{\ell{0}}=& \frac{x_{max}}{v_{x}}\\
%     \mathbf{t}_{\ell{1}}=& 1\\
%     \mathbf{t}_{\ell{2}}=& \frac{x_{max}}{v_{x}}\\
%     \mathbf{t}_{\ell{3}}=& 10.
% \end{align*}
% based on the continuous behaviour of the variables, the guard and invariance conditions. As there is no continuous flow in $\ell_{3}$ we set an arbitrary time out.
%
% With $Q_{A}$ and $\mathbb{T}$ we can now find the LTS abstraction of our PHA, to be
% \begin{itemize}
%     \item $Q=Q_{A}$
%     \item $Q_{0}=\{(\ell_{0},0,0,0)\}$
%     \item $A=\{t(\ell_{0}),t(\ell_{1}),t(\ell_{2}),\bot\}$
%     \item
% \end{itemize}
\end{ex}
