%%%%%%%%%%%%%%%%%%%%%%%%%%%%%%%%%%%%%%%%%%%%%%%%%%%%%%%%%%%%%%%%%%%%%%%%%%%%%%%%%%%%%%%%%%%%%%%%%%%
%%%%%%%%%%%%%%%%%%%%%%%%%%%%%%%%%%%%%%%%%%%%%%%%%%%%%%%%%%%%%%%%%%%%%%%%%%%%%%%%%%%%%%%%%%%%%%%%%%%
%%% PROB HA
%%%%%%%%%%%%%%%%%%%%%%%%%%%%%%%%%%%%%%%%%%%%%%%%%%%%%%%%%%%%%%%%%%%%%%%%%%%%%%%%%%%%%%%%%%%%%%%%%%%
%%%%%%%%%%%%%%%%%%%%%%%%%%%%%%%%%%%%%%%%%%%%%%%%%%%%%%%%%%%%%%%%%%%%%%%%%%%%%%%%%%%%%%%%%%%%%%%%%%%
\subsection{Probabilistic Hybrid Automata}
There are two ways of introducing stochastic behaviour into hybrid systems. The first type of hybrid system, a probabilistic hybrid automaton (PHA), has probabilistic discrete transitions between the locations. The other class of hybrid systems, stochastic hybrid automata (SHA), contain stochastic behaviour within the continuous flows. We will first investigate PHAs before looking at different types of SHAs.

Let us introduce the necessary notation from probability theory.
Let $S$ be a set. We define a \emph{distribution} $p: S\rightarrow [0,1]$ over $S$ to be such that $p(s)>0$ for at most a countable number of elements $s\in S$ and $\sum_{s\in S}p(s)=1$. We denote the set of distributions over $S$ by $\mu(S)$.
A distribution $p(S)$ is \emph{discrete} if the set $S$ is countable.
The \emph{Diract distribution} $p$ is a distribution defined over a singleton set $S=\{s\}$, such that $p(s)=1$.
The \emph{support} of a distribution $p(S)$ is defined as $support(p)=\{s\in S : p(s) >0\}$.

\begin{defi}[PHA \cite{Hahn2011}]
A \emph{probabilistic hybrid automaton} (PHA) $\mathcal{H} = (V,L,A,\Init,\Inv,E,f,t)$ consists of a hybrid automaton $\mathcal{A}=(V,L,A,\Init,\Inv,E,f)$ and a \emph{probabilistic transitions function} defined on the edges between the locations, $t : L\times A \rightarrow \mu(L)$, where the guard and reset maps, $g(\ell,a,\ell'),\ r(\ell,a,\ell')$  are defined over the set of edges $E \subseteq L \times A \times L$ if $t(\ell,a)=p(\ell')\neq 0$.
\end{defi}

Consider $p\in\mu(L)$, where $support(p)=\{\ell_{1},\ldots,\ell_{n}\}$. For a valuation $v\in\val(V)$, $Bundle(v,p)\subseteq \val(V)^n$ denotes the largest set of vectors of valuations such that $[v_{1},\ldots,v_{n}]\in Bundle(v,p)$ implies that $v_{i}\in r(e)_{i}$ (for $e=(\ell,a,\ell')$) for each $1\leq i\leq n$.

\begin{ex}[Probabilistic Hybrid Automaton \cite{Sproston2000}]
Figure~\ref{fig:exPHA} shows a simple probabilistic hybrid automaton describing the action of repeatedly transferring data packets from a sender to a receiver, with the possibility of data loss.
\begin{figure}[H]
    \begin{center}
        \begin{tikzpicture}[node distance=4cm]%, initial text = {$x,y=0$}]
            \node [state, initial] (s0) {$l_{0}$ \\ $\frac{9}{10}\leq\dot{x}\leq\frac{11}{10}$ \\ $x\leq5$};
            \node [state] (s1) [below left of = s0,xshift=-1cm] {$l_{1}$ \\ $\frac{4}{5}\leq\dot{x}\leq\frac{6}{5}$ \\ $x\leq 5$};
            \node [state] (s2) [below right of = s0,xshift=1cm] {$l_{2}$ \\ $\dot{x}=0$};

            \path[->] (s0) edge [bend left=10] node [right] {$0.99:($sent, $x\geq=4,x := 0)$} (s1)
                        edge node [right] {$0.01:($sent, $x\geq4)$} (s2)
                    (s1) edge [bend left=10] node [left] {$1:($received, $x=5)$} (s0)
                    (s2) edge [loop right] node [below,yshift=-2mm,xshift=5mm] {$1:($lost, $x:=0)$} (s2);
        \end{tikzpicture}
        \caption{Example of a probabilistic hybrid automaton.}
        \label{fig:exPHA}
    \end{center}
\end{figure}
\end{ex}

In the following sections we will introduce different classes of PHA, which have the same restrictions in their components as the HAs discussed in Section~\ref{sec:hybrid}.

%%%%%%%%%%%%%%%%%%%%%%%%%%%%%%%%%%%%%%%%%%%%%%%%%%%%%%%%%%%%%%%%%%%%%%%%%%%%%%%%%%%%%%%%%%%%%%%%%%%
%%%%%%%%%%%%%%%%%%%%%%%%%%%%%%%%%%%%%%%%%%%%%%%%%%%%%%%%%%%%%%%%%%%%%%%%%%%%%%%%%%%%%%%%%%%%%%%%%%%
%%% RECTANGULAR HA
%%%%%%%%%%%%%%%%%%%%%%%%%%%%%%%%%%%%%%%%%%%%%%%%%%%%%%%%%%%%%%%%%%%%%%%%%%%%%%%%%%%%%%%%%%%%%%%%%%%
%%%%%%%%%%%%%%%%%%%%%%%%%%%%%%%%%%%%%%%%%%%%%%%%%%%%%%%%%%%%%%%%%%%%%%%%%%%%%%%%%%%%%%%%%%%%%%%%%%%
\subsubsection{Probabilistic Rectangular Hybrid Automata}
\begin{defi}[Rectangular PHA \cite{Sproston2001}]
Let $\mathcal{H}=(V,L,A,\Init,\Inv,E,f,t)$ be a probabilistic hybrid automaton. Then $\mathcal{H}$ is a \emph{probabilistic rectangular hybrid automaton} if
\begin{itemize}
    \item{for every $\ell\in L$ the sets $\Init(\ell)$ and $\Inv(\ell)$ are rectangular,}
    \item{for every $\ell\in L$, there is a rectangular set $R^{\ell}$ such that $f(\ell)(x)=R^{\ell}$,}
    \item{and on every edge $e\in E$ the guard set $g(e)$, and reset map $r(e)$ are rectangular.}
\end{itemize}
\end{defi}

\begin{ex}[Probabilistic Rectangular Hybrid Automaton \cite{Sproston1999}]
Figure~\ref{fig:exPRHA} considers a conveyor belt in a production line, where an object moves forward over two sections, one moving faster than the other. Should a fault occur, the whole system is halted for a given time. The job is seen as finished, and a new object can be added to the conveyor belt, if the first object reaches the end.
\begin{figure}[H]
    \begin{center}
        \begin{tikzpicture}[node distance=3.5cm]%, initial text = {$x,y=0$}]
            \node [state, initial] (s0) {$l_{0}$ \\ $\dot{d}=0,\ \dot{x}=0$};
            \node [state] (s1) [right of = s0,node distance=6.5cm] {$l_{1}$ \\ $\dot{d}=0,$\\$2\leq\dot{x}\leq3$ \\ $x\leq 3$};
            \node [state] (s2) [below of = s1] {$l_{2}$ \\ $\dot{d}=1,\ \dot{x}=0$ \\ $d\leq 2$};
            \node [state] (s3) [below of = s0] {$l_{3}$ \\ $\dot{d}=0,$\\$1\leq\dot{x}\leq4$ \\ $x\leq6$};

            \path[->] (s0) edge node [above] {$1:($start, $d,x := 0)$} (s1)
                    (s1) edge node [right] {$0.01:($move, $x=3,d:=0,x:=3)$} (s2)
                        edge node [left,yshift=5mm,xshift=7mm] {$0.99:($move, $x=3, x:=3)$} (s3)
                    (s2) edge node [below] {$1:($move, $d=2,d:=0,x:=3)$} (s3)
                    (s3) edge node [left] {$1:($done, $x=6,x:=0)$} (s0);
        \end{tikzpicture}
        \caption{Example of a probabilistic rectangular hybrid automaton.}
        \label{fig:exPRHA}
    \end{center}
\end{figure}
\end{ex}

%%%%%%%%%%%%%%%%%%%%%%%%%%%%%%%%%%%%%%%%%%%%%%%%%%%%%%%%%%%%%%%%%%%%%%%%%%%%%%%%%%%%%%%%%%%%%%%%%%%
%%%%%%%%%%%%%%%%%%%%%%%%%%%%%%%%%%%%%%%%%%%%%%%%%%%%%%%%%%%%%%%%%%%%%%%%%%%%%%%%%%%%%%%%%%%%%%%%%%%
%%% MULTIRATE HA
%%%%%%%%%%%%%%%%%%%%%%%%%%%%%%%%%%%%%%%%%%%%%%%%%%%%%%%%%%%%%%%%%%%%%%%%%%%%%%%%%%%%%%%%%%%%%%%%%%%
%%%%%%%%%%%%%%%%%%%%%%%%%%%%%%%%%%%%%%%%%%%%%%%%%%%%%%%%%%%%%%%%%%%%%%%%%%%%%%%%%%%%%%%%%%%%%%%%%%%
\subsubsection{Probabilistic Multisingular Hybrid Automata}
\begin{defi}[Multisingular PHA \cite{Sproston2001}]
A \emph{probabilistic multisingular hybrid automaton} $\mathcal{H}=(V,L,A,\Init,\Inv,E,f,t)$ is a probabilistic rectangular hybrid automaton with the additional restriction that for each location $\ell\in L$ we have that $f(\ell)=\{\dot{x}=k : \forall x\in V\}$ for some $k\in\mathbb{N}$.
\end{defi}

%\begin{ex}
%Prob MS HA example goes here.
%\end{ex}
%%%%%%%%%%%%%%%%%%%%%%%%%%%%%%%%%%%%%%%%%%%%%%%%%%%%%%%%%%%%%%%%%%%%%%%%%%%%%%%%%%%%%%%%%%%%%%%%%%%
%%%%%%%%%%%%%%%%%%%%%%%%%%%%%%%%%%%%%%%%%%%%%%%%%%%%%%%%%%%%%%%%%%%%%%%%%%%%%%%%%%%%%%%%%%%%%%%%%%%
%%% STOPWATCH Aut
%%%%%%%%%%%%%%%%%%%%%%%%%%%%%%%%%%%%%%%%%%%%%%%%%%%%%%%%%%%%%%%%%%%%%%%%%%%%%%%%%%%%%%%%%%%%%%%%%%%
%%%%%%%%%%%%%%%%%%%%%%%%%%%%%%%%%%%%%%%%%%%%%%%%%%%%%%%%%%%%%%%%%%%%%%%%%%%%%%%%%%%%%%%%%%%%%%%%%%%

\subsubsection{Probabilistic Stopwatch Automata}
\begin{defi}[Probabilistic Stopwatch Automaton \cite{Sproston2001}]
A \emph{Probabilistic Stopwatch Automaton} is a probabilistic multisingular hybrid automaton where for each location $\ell\in L$ we have that $f(\ell)=\{\dot{x}=0 \text{ or } \dot{x}=1 : x\in V\}$.
\end{defi}

%\begin{ex}
%Prob SW Aut example goes here.
%\end{ex}

%%%%%%%%%%%%%%%%%%%%%%%%%%%%%%%%%%%%%%%%%%%%%%%%%%%%%%%%%%%%%%%%%%%%%%%%%%%%%%%%%%%%%%%%%%%%%%%%%%%
%%%%%%%%%%%%%%%%%%%%%%%%%%%%%%%%%%%%%%%%%%%%%%%%%%%%%%%%%%%%%%%%%%%%%%%%%%%%%%%%%%%%%%%%%%%%%%%%%%%
%%% TIMED Aut
%%%%%%%%%%%%%%%%%%%%%%%%%%%%%%%%%%%%%%%%%%%%%%%%%%%%%%%%%%%%%%%%%%%%%%%%%%%%%%%%%%%%%%%%%%%%%%%%%%%
%%%%%%%%%%%%%%%%%%%%%%%%%%%%%%%%%%%%%%%%%%%%%%%%%%%%%%%%%%%%%%%%%%%%%%%%%%%%%%%%%%%%%%%%%%%%%%%%%%%
\subsubsection{Probabilistic Timed Automata}
\begin{defi}[PTA]
\label{def:PTA}
Let $\mathcal{H}=(V,L,A,\Init,\Inv,E,f,t)$ be a probabilistic hybrid automaton. Then $\mathcal{H}$ is a \emph{probabilistic timed automaton} if the flow condition in all locations is of the form $\dot{x}=1$, $\forall x\in V$.
\end{defi}

A probabilistic timed automaton is a probabilistic stopwatch automaton with the additional restriction that for each location the flow of all continuous variables is $\dot{x}=1$.

\begin{ex}[Communication Protocol \cite{Sproston2001}]
    This example models a communication protocol between a sender \textsf{s} and a receiver \textsf{r}. When \textsf{s} has some data to transmit, it sends it over to \textsf{r}. If the transmission was successful, \textsf{r} transmits an acknowledgement. There is a timeout to the transmission of the data from \textsf{s} to \textsf{r}, should the data not arrive with \textsf{r} within that time, the whole process is aborted. Figure~\ref{fig:excomm} illustrates a hybrid automaton of this protocol.
\begin{figure}[H]
    \begin{center}
        \begin{tikzpicture}[node distance=3cm]%, initial text = {$x,y=0$}]
            \node [state, initial] (s0) {\textsf{s}=hasData,\\ \textsf{r}=idle \\ $\dot{x}=1,\ \dot{y}=1$ \\ $x\leq 3$};
            \node (A) [below of = s0] {};
            \node [state] (s1) [left of = A,node distance=4cm] {\textsf{s}=wait,\\ \textsf{r}=received \\ $\dot{x}=1,\ \dot{y}=1$ \\ $y\leq 1$};
            \node [state] (s2) [right of = A,node distance=4cm] {\textsf{s}=wait,\\\textsf{r}=idle \\ $\dot{x}=1,\ \dot{y}=1$ \\ $x\leq 3,\ y\leq 7$};
            \node [state] (s3) [below of = s1] {\textsf{s}=received,\\\textsf{r}=idle \\ $\dot{x}=1,\ \dot{y}=1$};
            \node [state] (s4) [below of = s2] {\textsf{s}=abort,\\\textsf{r}=abort \\ $\dot{x}=1,\ \dot{y}=1$};

            \path[->] (s0) edge node [left] {$0.95:($send, $x\geq 2,\ x,y := 0)$} (s1)
                        edge node [right] {$0.05:($send, $x\leq 2,\ x:=0)$} (s2)
                    (s1) edge [bend right=15] node [below] {$0.01:($true)} (s2)
                        edge node [left] {$0.99:($true)} (s3)
                    (s2) edge node [above,xshift=8mm] {$0.95:($waiting, $x\geq 2,\ x,y := 0)$} (s1)
                        edge [loop right] node [above,yshift=3mm,xshift=2mm] {$0.05:($waiting, $x:=0$)} (s2)
                        edge node [right] {$1:($abort, $y=7,\ x,y := 0)$} (s4)
                    (s3) edge node [below,yshift=-1.5cm] {$1:($newdata)} (s0)
                    (s4) edge [loop right] node [below,yshift=-4mm] {$1:($quit, $x,y := 0$)} (s4);
        \end{tikzpicture}
        \caption{Example of a communication protocol as a probabilistic timed automaton.}
        \label{fig:excomm}
    \end{center}
\end{figure}
\end{ex}


%%%%%%%%%%%%%%%%%%%%%%%%%%%%%%%%%%%%%%%%%%%%%%%%%%%%%%%%%%%%%%%%%%%%%%%%%%%%%%%%%%%%%%%%%%%%%%%%%%%
%%%%%%%%%%%%%%%%%%%%%%%%%%%%%%%%%%%%%%%%%%%%%%%%%%%%%%%%%%%%%%%%%%%%%%%%%%%%%%%%%%%%%%%%%%%%%%%%%%%
%%% O MIN HA
%%%%%%%%%%%%%%%%%%%%%%%%%%%%%%%%%%%%%%%%%%%%%%%%%%%%%%%%%%%%%%%%%%%%%%%%%%%%%%%%%%%%%%%%%%%%%%%%%%%
%%%%%%%%%%%%%%%%%%%%%%%%%%%%%%%%%%%%%%%%%%%%%%%%%%%%%%%%%%%%%%%%%%%%%%%%%%%%%%%%%%%%%%%%%%%%%%%%%%%
\subsubsection{Probabilistic O-Minimal Hybrid Automata}
\begin{defi}[O-Minimal PHA]
A probabilistic hybrid system $\mathcal{H}=(V,L,A,\Init,\Inv,E,f,t)$ is called \emph{o-minimal} if
\begin{itemize}
    \item{for each $\ell\in L$, the flow $f(\ell)$ is a complete differential function (defined for all time);}
    \item{for each $e\in E$, the reset map $r(e)$ is a piecewise constant but set valued map, with a finite number of pieces;}
    \item{for each $\ell\in L$ and all $e\in E$, $\Inv(\ell),\ \Init(\ell),\ , g(e),\ f(\ell)$ are definable over the same o-minimal theory over $\mathbb{R}$.}
\end{itemize}
\end{defi}

%\begin{ex}
%Prob o-min HA example goes here.
%\end{ex}

%%%%%%%%%%%%%%%%%%%%%%%%%%%%%%%%%%%%%%%%%%%%%%%%%%%%%%%%%%%%%%%%%%%%%%%%%%%%%%%%%%%%%%%%%%%%%%%%%%%
%%%%%%%%%%%%%%%%%%%%%%%%%%%%%%%%%%%%%%%%%%%%%%%%%%%%%%%%%%%%%%%%%%%%%%%%%%%%%%%%%%%%%%%%%%%%%%%%%%%
%%% STORMED HA
%%%%%%%%%%%%%%%%%%%%%%%%%%%%%%%%%%%%%%%%%%%%%%%%%%%%%%%%%%%%%%%%%%%%%%%%%%%%%%%%%%%%%%%%%%%%%%%%%%%
%%%%%%%%%%%%%%%%%%%%%%%%%%%%%%%%%%%%%%%%%%%%%%%%%%%%%%%%%%%%%%%%%%%%%%%%%%%%%%%%%%%%%%%%%%%%%%%%%%%
\subsubsection{Probabilistic STORMED Hybrid Automata}
\begin{defi}[Probabilistic STORMED Hybrid System]
A \emph{probabilistic STORMED hybrid system} is a tuple $(\mathcal{H},\mathcal{S},x,b_{-},b_{+},d_{min},\epsilon,\zeta)$ where $\mathcal{H}$ is a probabilistic hybrid automaton, $\mathcal{S}$ is an o-minimal structure, $b_{-},b_{+},d_{min}\in\mathbb{R}$ and $x\in \val(V)$ is a vector such that the following are satisfied:
\begin{itemize}
    \item the guards of $\mathcal{H}$ are $d_{min}$-\textbf{S}eparable;
    \item the flows of $\mathcal{H}$ are \textbf{T}ICS;
    \item $\mathcal{H}$ is definable in the \textbf{O}-minimal structure $\mathcal{S}$;
    \item \textbf{R}esets and flows are \textbf{M}onotonic, $(\epsilon,\zeta,x)$-monotonic and $(\epsilon,x)$-monotonic respectively.
    \item \textbf{E}nds are \textbf{D}elimited: $\forall e=(\ell,a,\ell')\in E$ we have
    $\{val(x)\cdot y : y\in g(e)\in(b_{-},b_{+})$ meaning that the projection of each of the guard conditions on $val(x)$ lies in the open interval $(b_{-},b_{+})$.
\end{itemize}
\end{defi}

%\begin{ex}
%Prob STORMED HA example goes here.
%\end{ex}
%%%%%%%%%%%%%%%%%%%%%%%%%%%%%%%%%%%%%%%%%%%%%%%%%%%%%%%%%%%%%%%%%%%%%%%%%%%%%%%%%%%%%%%%%%%%%%%%%%%
%%%%%%%%%%%%%%%%%%%%%%%%%%%%%%%%%%%%%%%%%%%%%%%%%%%%%%%%%%%%%%%%%%%%%%%%%%%%%%%%%%%%%%%%%%%%%%%%%%%
%%% STOCHASTIC HA
%%%%%%%%%%%%%%%%%%%%%%%%%%%%%%%%%%%%%%%%%%%%%%%%%%%%%%%%%%%%%%%%%%%%%%%%%%%%%%%%%%%%%%%%%%%%%%%%%%%
%%%%%%%%%%%%%%%%%%%%%%%%%%%%%%%%%%%%%%%%%%%%%%%%%%%%%%%%%%%%%%%%%%%%%%%%%%%%%%%%%%%%%%%%%%%%%%%%%%%
\subsection{Stochastic Hybrid Automata}
\label{sec:sha}
We now move to the class of hybrid systems containing stochastic behaviour in the continuous flow. For that we need to redefine the hybrid automaton slightly. First let us introduce some notation from stochastic theory.
A family $\Sigma$ of subsets of an arbitary set $\Omega$ (the \emph{sample space}) is a \emph{$\sigma$-algebra}, if $\Omega\in\Sigma$ and $\Sigma$ is closed under the complement and countable unions.
A set $A\in\Sigma$ is called \emph{measurable}.
The pair $(\Omega,\Sigma)$ is called a \emph{measurable space}.
Given a family of sets $B$ we denote by $\sigma(B)$ the \emph{$\sigma$-algebra generated} by $B$, which is the smallest $\sigma$-algebra containing $B$.
Let $(\Omega_{1},\Sigma_{1})$ and $(\Omega_{2},\Sigma_{2})$ be two measurable spaces. We say that the function $f:\Omega_{1}\rightarrow\Omega_{2}$ is \emph{$\Sigma_{1}-\Sigma_{2}$-measurable} if $f^{-1}(B)\in\Sigma_{1}$ for all $B\in\Sigma_{2}$.
The \emph{Borel $\sigma$-algebra} over $\Omega$ is generated by the open subsets of $\Omega$ and is denoted by $\mathcal{B}(\Omega)$.
A function $f:\Sigma\rightarrow\mathbb{R}_{\geq0}$ is called a \emph{measure} if $f(\emptyset)=0$ and for a countable number of disjoint sets $B_{i}$, $f(\bigcup_{i}B_{i}) = \sum_{i}f(B_{i})$.
A function $f$ is called a \emph{probability measure} if it is a measure and $f(\Omega)=1$.
We denote by $\Delta(\Omega)$ the set of all probability measures in the measurable space. It can be equipped with the $\sigma$-algebra $\Delta(\Sigma)$ (a set of sets of probability measures)  generated by the measures which when applied to $S\in\Sigma$ give a value $B\in\mathcal{B}([0,1])$
\[
\Delta(\Sigma) = \sigma(\{\Delta^{B}(S): S\in\Sigma\text{ and } B\in\mathcal{B}([0,1])\})
\]
where $\Delta^{B}(S)=\{p\in\Delta(\Sigma):p(S)\in B\}$.
Let $\Sigma$ be a $\sigma$-algebra, then the \emph{hit $\sigma$-algebra} over $\Sigma$ is
\[
H(\Sigma) = \sigma(\{H_{S} : S\in\Sigma\})
\]
where $H_{S} = \{ C\in \Sigma : C\cap S \neq \emptyset\}$.
Let $(\Omega_{1},\Sigma_{1})$ and $(\Omega_{2},\Sigma_{2})$ be two measurable spaces. Then a \emph{measurable rectangle} is s subset $S_{1}\times S_{2}$ of $\Omega_{1}\times\Omega_{2}$ where
 $S_{1}\in\Sigma_{1}$ and $S_{2}\in\Sigma_{2}$ are measurable subsets of $\Omega_{1}$ and $\Omega_{2}$ respectively.
Let $(\Omega_{1},\Sigma_{1})$ and $(\Omega_{2},\Sigma_{2})$ be two measurable spaces. Then the \emph{product $\sigma$-algebra} $\Sigma_{1}\otimes\Sigma_{2}$ is defines as the $\sigma$-algebra on $\Omega_{1}\times\Omega_{2}$ generated by the collection of all measurable rectangles,
\[
\Sigma_{1}\otimes\Sigma_{2} = \sigma(\{S_{1}\times S_{2} : S_{1}\in\Sigma_{1}, S_{2}\in\Sigma_{2}\}).
\]

\begin{defi}[SHA \cite{Hahn2012}]
A \emph{stochastic hybrid automaton} (SHA) $\mathcal{S}=(V, L, \Init, f, \tfin, \tcont)$ consists of

\begin{description}
    \item[$V$] a set of $k$ \emph{continuous variables}, the valuation of $\val(V)\subseteq\mathbb{R}^{k}$;
    \item[$L$] a finite set of \emph{locations/modes/discrete variables};
    \item[$\Init\in L$] is the initial location;
    \item[$f: L \rightarrow (\val(V)\times \mathbb{R}_{\geq0} \rightarrow 2^{\mathbb{R}^k})$] the flow function of the continuous variables, where for each $\ell\in L$ $f(\ell)=F$ is, for each $x\in\val(V)$
    \begin{itemize}
        \item $F(x,0)=x$,
        \item $\forall t,t'\in \mathbb{R}_{\geq0}$ with $0\leq t'\leq t$ we have $F(x,t)=\bigcup_{x'\in F(x,t')} F(x',t-t')$,
        \item $\exists t\in \mathbb{R}_{\geq0}$ where $\forall t'\geq t$ $F(x,t')=F(x,t)$ and $\forall x'\in F(x,t')$ and all $t''\in \mathbb{R}_{\geq0}$, $F(x',t'')=\{v'\}$;
    \end{itemize}
    \item[$\tfin$] is the set of measurable finite guarded transitions;
    \item[$\tcont$] is the set of measurable continuous guarded transitions.
\end{description}
For each state $(\ell,x)\in L\times\val(V)$ with $f(\ell)(x,t)=\{x\}$, $\forall t\in\mathbb{R}_{\geq0}$ there is a (finite or continuous) transition with guard $g$ with $(\ell,x)\in g$.
\end{defi}

\begin{defi}[Measurable Guarded Transition]
Let $k\in\mathbb{N}$, $L$ be a finite set of locations, $Q=L\times\val(V)$, $\Sigma_{Q}=2^{L}\otimes\mathcal{B}(\val(V))$.

A \emph{measurable finite guarded transition} is of the form
\[
    g \rightarrow p_{1}:u_{1}+\ldots+p_{n}:u_{n}
\]
where $g$ is a guard, $p_{i}\geq 0$ for $0\leq i\leq n$, $\sum_{i}p_{i}=1$, $u_{i}:Q\rightarrow \Sigma_{Q}$ are the update functions, and are $\Sigma_{Q}-H(\Sigma_{Q})$-measurable, $u_{i}(q)\neq\emptyset$ for $1\leq i\leq n$ if $q\in g$.

A \emph{measurable continuous guarded transition} is of the form
\[
    g \rightarrow L
\]
where $g$ is defined as for finite guarded transitions and $L: Q \rightarrow \Delta(Q)$ is a $\Sigma_{Q}-\Delta(\Sigma_{Q})$-measurable function.
\end{defi}

This class of hybrid automata allows for continuous stochastic flow as well as stochastic jumps, examples for different specific types of stochastic flow can be found in \cite{Koutsoukos2006} where it is described as a Wiener process, or in \cite{Hu2000} the flow is a standard Brownian motion.

\begin{ex}[Stochastic Thermostat \cite{Hahn2012}]
    Figure~\ref{fig:shathem} shows an example of an SHA, where we are looking at a thermostat. It starts heating up a room, while every time step $t=1$, it measures the temperature of the room $m:=\mathcal{N}(T,0.25)$ which is modelled as  a normal distribution with standard deviation 0.25. If the temperature reaches a limit, the thermostat either starts cooling the room, $\dot{T}=-T$, or warming it, $\dot{T}=2$.
\begin{figure}[H]
    \begin{center}
        \begin{tikzpicture}[node distance=6cm]%, initial text = {$x,y=0$}]
            \node [state,initial] (s0) {$\ell_{0}$\\$\dot{T}=2,$\\$\dot{t}=1$\\$t\leq 1$};
            \node [state] (s1) [right of = s0] {$\ell_{1}$\\ $\dot{T}=2,$\\ $\dot{t}=1$,\\ $t\leq0.1$};
            \node [state] (s2) [below of = s0,node distance=4cm] {$\ell_{2}$\\ $\dot{T}=-T,$\\ $\dot{t}=1$\\ $t\leq 1$};
            \node [state] (s3) [right of = s2] {$\ell_{3}$\\ $\dot{T}=-T,$\\ $\dot{t}=1$,\\ $t\leq0.1$};

            \path[->] (s0) edge [bend left=12] node [above] {$t\geq1$, $t:=0,m:=\mathcal{N}(T,0.25)$} (s1)
                (s1) edge [bend left=12] node [below] {$t\geq 0.1, m<9$, $t:=0$} (s0)
                    edge node [xshift=2.8cm,yshift=0.8cm] {$t\geq 0.1, m\geq9$, $t:=0$} (s2)
                (s2) edge [bend right=12] node [below] {$t\geq1$, $t:=0,m:=\mathcal{N}(T,0.25)$} (s3)
                (s3) edge node [xshift=2.8cm,yshift=-0.8cm] {$t\geq 0.1, m\leq6$, $t:=0$} (s0)
                    edge [bend right=12] node [above] {$t\geq 0.1, m>6$, $t:=0$} (s2);

        \end{tikzpicture}
        \caption{Example of a thermostat modeled as a stochastic hybrid automaton.}
         \label{fig:shathem}
     \end{center}
\end{figure}
\end{ex}

% \begin{ex}[Probabilistic Lawn-mower \cite{Hahn2012}]
%     \begin{figure}[H]
%         \begin{center}
%             \begin{tikzpicture}[node distance=2.5cm]%, initial text = {$x,y=0$}]
%                 \node [state, initial] (NE1) {NE1\\$\dot{x}=v_{x}$\\$\dot{y}=v_{y}$\\$x\leq l,y\leq h$};
%                 \node [state] (NE2) [right of = NE1] {NE2\\$\dot{x}=v_{x}'$\\$\dot{y}=v_{y}'$\\$x\leq l,y\leq h$};
%
%                 \node [state] (NW1) [right of = NE2,node distance=3cm] {NW1\\$\dot{x}=-v_{x},\dot{y}=v_{y}$\\$x\geq 0,y\leq h$};
%                 \node [state] (NW2) [right of = NW1] {NW2\\$\dot{x}=-v_{x}',\dot{y}=v_{y}'$\\$x\geq 0,y\leq h$};
%
%                 \node [state] (SE1) [below of = NE1,node distance=3cm] {SE1\\$\dot{x}=v_{x},\dot{y}=-v_{y}$\\$x\leq l,y\geq 0$};
%                 \node [state] (SE2) [right of = SE1] {SE2\\$\dot{x}=v_{x}',\dot{y}=-v_{y}'$\\$x\leq l,y\geq 0$};
%
%                 \node [state] (SW1) [right of = SE2,node distance=3cm] {SW1\\$\dot{x}=-v_{x},\dot{y}=-v_{y}$\\$x\geq 0,y\geq 0$};
%                 \node [state] (SW2) [right of = SW1] {SW2\\$\dot{x}=-v_{x}',\dot{y}=-v_{y}'$\\$x\geq l,y\geq 0$};
%
%
%
%
%             \end{tikzpicture}
%             \caption{Example of a probabilistic lawn-mower modelled as a stochastic hybrid automaton.}
%             \label{fig:exlawn}
%         \end{center}
%     \end{figure}
% \end{ex}
%%%%%%%%%%%%%%%%%%%%%%%%%%%%%%%%%%%%%%%%%%%%%%%%%%%%%%%%%%%%%%%%%%%%%%%%%%%%%%%%%%%%%%%%%%%%%%%%%%%
%%%%%%%%%%%%%%%%%%%%%%%%%%%%%%%%%%%%%%%%%%%%%%%%%%%%%%%%%%%%%%%%%%%%%%%%%%%%%%%%%%%%%%%%%%%%%%%%%%%
%%% DISC TIME STOCH HA
%%%%%%%%%%%%%%%%%%%%%%%%%%%%%%%%%%%%%%%%%%%%%%%%%%%%%%%%%%%%%%%%%%%%%%%%%%%%%%%%%%%%%%%%%%%%%%%%%%%
%%%%%%%%%%%%%%%%%%%%%%%%%%%%%%%%%%%%%%%%%%%%%%%%%%%%%%%%%%%%%%%%%%%%%%%%%%%%%%%%%%%%%%%%%%%%%%%%%%%
\subsection{Discrete Time Stochastic Hybrid Systems}
Another class of SHA are discrete time stochastic hybrid systems. These SHA combine both the stochastic behaviour of the continuous flow inside the locations and the probabilistic discrete transitions between locations. The whole system evolves over a finite time horizon or over an infinite countable time horizon.

One further definition from stochastic theory is needed before we can define discrete time stochastic hybrid systems.
Let $(\Omega_{1},\Sigma_{1})$ and $(\Omega_{2},\Sigma_{2})$ be two measurable spaces. A \emph{stochastic kernel} from $(\Omega_{1},\Sigma_{1})$ to $(\Omega_{2},\Sigma_{2})$ is the function
$k: \Omega_{1}\times\Sigma_{2}\rightarrow[0,1]$ such that the following hold
\begin{itemize}
    \item $x\mapsto k(x,S_{2})$ is $\Sigma_{1}$-measurable for every $S_{2}\in\Sigma_{2}$
    \item $S_{2}\mapsto k(x,S_{2})$ is a probability measure on $(\Omega_{2},\Sigma_{2})$ for every $x\in\Omega_{1}$.
\end{itemize}

\begin{defi}[DTSHS \cite{Amin2006,Abate2010}]
\label{def:DTSHS}
A \emph{discrete time stochastic hybrid system} (DTSHS) $\mathcal{D}=(L,n,\Init,T_x,T_q,R)$ is defined as
\begin{itemize}
    \item $L=\{\ell_{1},\ldots,\ell_{m}\}$, $m\in\mathbb{N}$ is the discrete state space/locations/modes;
    \item $n:L\rightarrow\mathbb{N}$ assigns to each $\ell\in L$ the dimension of the continuous state space $\mathbb{R}^{n(\ell)}$;

    \item $\Init:\mathcal{B}(Q)\rightarrow [0,1]$ is a probability measure on
     $Q=\bigcup_{\ell\in L} \{\ell\}\times\mathbb{R}^{n(\ell)}$ for the initial state;

    \item $T_x:\mathcal{B}(\mathbb{R}^{n(\cdot)})\times Q \rightarrow [0,1]$ is a conditional stochastic kernel on $\mathbb{R}^{n(\cdot)}$ given $Q$, which assigns to each $q=(\ell,x)\in Q$ a probability measure $T_x(\cdot|s)$ on the Borel space $(\mathbb{R}^{n(\ell)},\mathcal{B}(\mathbb{R}^{n(\ell)}))$.
    The function $T_x(A|(\ell,\cdot))$ is assumed to be Borel measurable $\forall \ell\in L$ and $\forall A\in\mathcal{B}(\mathbb{R}^{n(\ell)})$;

    \item $T_q:L\times Q\rightarrow [0,1]$  is a conditional discrete stochastic kernel on $L$ given $Q$ which assigns to each $q\in Q$ a probability distribution $T_q(\cdot|s)$ over $L$;

    \item $R:\mathcal{B}(\mathbb{R}^{n(\cdot)})\times Q\times L\rightarrow[0,1]$ is a conditional stochastic kernel on $\mathbb{R}^{n(\cdot)}$ given $Q\times L$ that assigns to each $q\in Q$ and $\ell'\in L$ a probability measure $R(\cdot | s, \ell')$ on the Borel space
     $(\mathbb{R}^{n(\ell')},\mathcal{B}(\mathbb{R}^{n(\ell')}))$.The function $R(A|(\ell,\cdot),\ell')$  is assumed to be Borel measurable for all $\ell,\ell'\in L$ and all $A\in\mathcal{B}(\mathbb{R}^{n(\ell')})$.
\end{itemize}
\end{defi}
Note that $Q=\bigcup_{\ell\in L} \{\ell\}\times\mathbb{R}^{n(\ell)}$ is the state space.
We can simplify the various transition functions into one "larger" function $T:\mathcal{B}(Q)\times Q\rightarrow [0,1]$ which is the conditional stochastic kernel on $Q$ given $Q$ and is defined by
\[
T(\{\ell'\}\times A_{\ell}), (\ell,x)) =
\begin{cases}
T_{x}(A_{\ell'},(\ell,x))T_{q}(\ell', (\ell,x)) & \text{ if } \ell'=\ell \\
R(A_{\ell'},(\ell,x),\ell')T_{q}(\ell', (\ell,x)) & \text{ if } \ell'\neq\ell
\end{cases}.
\]

%\begin{ex}
%DTSHS example goes here.
%\end{ex}

%%%%%%%%%%%%%%%%%%%%%%%%%%%%%%%%%%%%%%%%%%%%%%%%%%%%%%%%%%%%%%%%%%%%%%%%%%%%%%%%%%%%%%%%%%%%%%%%%%%
%%%%%%%%%%%%%%%%%%%%%%%%%%%%%%%%%%%%%%%%%%%%%%%%%%%%%%%%%%%%%%%%%%%%%%%%%%%%%%%%%%%%%%%%%%%%%%%%%%%
%%% STOCH CONT HA
%%%%%%%%%%%%%%%%%%%%%%%%%%%%%%%%%%%%%%%%%%%%%%%%%%%%%%%%%%%%%%%%%%%%%%%%%%%%%%%%%%%%%%%%%%%%%%%%%%%
%%%%%%%%%%%%%%%%%%%%%%%%%%%%%%%%%%%%%%%%%%%%%%%%%%%%%%%%%%%%%%%%%%%%%%%%%%%%%%%%%%%%%%%%%%%%%%%%%%%
% \subsection{Stochastic Continuous Behaviour}
% Another type of stochastic hybrid automata is governed by stochastic behaviour of the continuous variables in each mode, rather than deterministic behaviour. So the flow of the continuous variables in the modes evolves through stochastic differential equations
% \[
% \dot{x} = f(\ell)(x) + dB_{t},
% \]
% where $B_{t}$ can be seen as the standard Brownian motion in $\mathbb{R}$ \cite{Hu2000}, or as a $\mathbb{R}^{n}$ valued Wiener process \cite{Koutsoukos2006}.
%
% Further, the transitions between the modes will be probabilistic.
%
% \begin{defi}[Stochastic Hybrid Automaton \cite{Hu2000}]
% A \emph{stochastic hybrid automaton} is a hybrid automaton $\mathcal{A}=(V,L,A,\Init,\Inv,E,f)$ where
% \begin{itemize}
%     \item{for each edge $e\in E$ the guard $g(e)$ is a measurable subset of $\delta \Inv$, or empty,}
%     \item{the reset map $r$ is changed to be $r : E \rightarrow 2^{\val(V)}$, such that is assigns each edge $(\ell,a,\ell')\in E$ a reset probability kernel on $\val(V)$ concentrated on $\Inv(\ell')$. Further, for any measurable set $M\subset \Inv(\ell')$, $r(e)(M)$ is a measurable function.}
% \end{itemize}
% \end{defi}
%
% \begin{ex}[Car Chase \cite{Hu2000}]
% Imagine two cars $x_{1}$ and $x_{2}$ on a straight road. Both cars drive in the same direction, where the task for $x_{1}$ is to catch up with $x_{2}$ (which is driving at a constant speed) up to given distance, keep following, if $x_{1}$ gets too close, $x_{1}$ breaks and falls back. The motion of both cars is stochastic, due to many unknown factors such as the road condition or the environment. We can abstract the stochastic behaviour away from the motion of $x_{2}$ as we are modeling the behaviour of $x_{1}$. The distance between the two cars is $\Delta x$. There are multiple distance thresholds we consider $d_{0}>d_{1}>d_{2}>d_{3}>0$ which are in order, post breaking distance, chasing distance, keeping distance and being too close. Figure~\ref{fig:excar} shows the three modes of the scenario, chasing, keeping up and breaking. We denote the probability of $x_{1}$ falling behind $x_{2}$ further than the the keeping up distance by $p$, so the probability of being too close to $x_{2}$ is $1-p$.
% \begin{figure}[H]
%     \begin{center}
%         \begin{tikzpicture}[node distance=5.5cm]
%             \node [state] (s0) {$l_{0}$\\ $\dot{x}_{1}=v_{1}+dB_{t},$\\$\dot{x}_{2}=v_{2}$\\ $\Delta x>d_{2}$};
%             \node [state] (s1) [right of = s0] {$l_{1}$\\ $\dot{x}_{1}=v_{2}+dB_{t},$\\$\dot{x}_{2}=v_{2}$\\ $d_{3}<\Delta x<d_{1}$};
%             \node [state] (s2) [below right of = s0,xshift=-1cm] {$l_{2}$\\ $\ddot{x}_{1}=-a_{1}(t),$\\$\dot{x}_{2}=v_{2}$\\ $\Delta x<d_{0}$};
%             \path[->] (s0) edge [bend left=15] node [above] {$1:($Keep up, $\Delta x=d_{2})$} (s1)
%                 (s1) edge [bend left=15] node [below] {$p:($Chase, $\Delta x=d_{1})$} (s0)
%                     edge node [right] {$1-p:($Break, $\Delta x=d_{3})$} (s2)
%                 (s2) edge node [left] {$1:($Chase, $\Delta x=d_{0})$} (s0);
%         \end{tikzpicture}
%         \caption{Example of a simplistic car chase as a stochastic hybrid automaton.}
%         \label{fig:excar}
%     \end{center}
% \end{figure}
% \end{ex}
